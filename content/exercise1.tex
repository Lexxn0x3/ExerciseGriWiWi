\chapter{BWL}
\exercise{1\textbar 1}{Betriebswirtschaftslehre und Volkswirtschaftslehre}
Welche Gemeinsamkeiten und welche Unterschiede haben die Betriebswirtschaftslehre und die Volkswirtschaftslehre?

\solution{
\textbf{Gemeinsamkeiten}
\begin{itemize}
  \item Beide \textit{Wirtschaftswissenschaften}
  \item Real-, Geistes- und Sozialwissenschaft
  \item \textit{iterdisziplinäre Wissenschaften} -> Kentnisse andere Wissenschaften (Mathematik, Statistik)
  \item \textit{Problem der knappen Ressourcen}, ökonomischer umgang mit knappen Gütern (Optimierung)
\end{itemize}

\textbf{Unterschiede}
\begin{itemize}
  \item \textbf{BWL} Perspektive des \textit{einzelnen Unternehmens}, versucht unternehmens Prozesse zu analysieren und Handlungsempfehlungen für betriebliche Entscheidungen zu geben.
  \item \textbf{VWL} Untersucht \textit{Wirtschaftsablauf} und seine \textit{Gesetzmäßigkeiten}. Mit Modellen wird versucht sie, das Verhalten von Haushalten und Unternehmen in Märkten zu beschreiben und zu erklären, um Handlungsempfehlungen für die \textit{Wirtschaftspolitik} abzuleiten.
\end{itemize}
}

\exercise{1\textbar 2}{Produktivität und Wirtschaftlichkeit}
Aus 10 kg Draht können 1.000 Schrauben hergestellt werden.
Der Wert des Drahtes beläuft sich auf 2 €/kg.
Der Wert einer Schraube beträgt 0,02 €.
\begin{enumerate}[label=(\alph*)]
  \item Wie hoch ist die Produktivität und wie hoch ist die Wirtschaftlichkeit des Einsatzes von 10 kg Draht zur Herstellung von 1.000 Schrauben?
  \item Sie erhalten den Auftrag, die Produktivität der Schraubenherstellung um 10 \% zu steigern. Welche Möglichkeiten haben Sie?
  \item Sie erhalten den Auftrag, die Wirtschaftlichkeit der Schraubenherstellung um 10 \% zu erhöhen. Welche Möglichkeiten stehen Ihnen zur Verfügung?\\ \note{Lösungshinweis: Variieren Sie zum einen die Mengen, zum anderen die Preise.}
\end{enumerate}

\solution{
\begin{enumerate}[label=(\alph*)]
  \item \[ prod = \frac{Ausbringungsmenge}{Einsatzmenge}= \frac{1.000 \, Schrauben}{10 \, kg Draht} = 100 \, Schrauben/kg Draht\] 
        \[ wirt = \frac{Ertrag}{Kosten}= \frac{1.000 \cdot 0,02 \, €/Schraube}{10 \cdot 2 \, €/kg Draht} = \frac{20}{20} = 1\] 
  \item Um die Produktivität um 10 \% zu steigern, kann die Ausbringungsmenge erhöht oder die Einsatzmenge reduziert werden.  
      \begin{itemize}
        \item \textbf{Erhöhung der Ausbringungsmenge:}  
          Die neue Produktivität soll 110 Schrauben/kg betragen: \[ 110 = \frac{Ausbringungsmenge}{10}\]
          Somit beträgt die neue Ausbringungsmenge:  
          \[Ausbringungsmenge = 110 \cdot 10 = 1.100 \, Schrauben\]
        \item \textbf{Reduktion der Einsatzmenge:}  
          Die neue Produktivität soll ebenfalls 110 Schrauben/kg betragen:  
          \[110 = \frac{1.000}{Einsatzmenge}\]
          Somit beträgt die neue Einsatzmenge:  
          \[Einsatzmenge = \frac{1.000}{110} \approx 9,09 \, kg Draht\]
      \end{itemize}
  \item Um die Wirtschaftlichkeit um 10 \% zu steigern, stehen folgende Möglichkeiten zur Verfügung:  
    \begin{itemize}
      \item Erhöhung der Anzahl der Schrauben von 1.000 auf 1.100 Stück:  
        \[Wirtschaftlichkeit = \frac{1.100 \cdot 0,02}{10 \cdot 2} = \frac{22}{20} = 1,1\]
      \item Verminderung der Menge des eingesetzten Drahtes:  
        \[Wirtschaftlichkeit = \frac{1.000 \cdot 0,02}{9,09 \cdot 2} = \frac{20}{18,18} \approx 1,1\]
      \item Erhöhung des Preises für die Schrauben auf 0,022 €/Schraube:  
        \[Wirtschaftlichkeit = \frac{1.000 \cdot 0,022}{10 \cdot 2} = \frac{22}{20} = 1,1\]
      \item Reduktion des Preises für den Draht auf 1,818 €/kg:  
        \[Wirtschaftlichkeit = \frac{1.000 \cdot 0,02}{10 \cdot 1,818} = \frac{20}{18,18} \approx 1,1.\]
    \end{itemize}
\end{enumerate}
}

\exercise{1\textbar 3}{Produktivität, Wirtschaftlichkeit und Rentabilität}
Das Unternehmen High Sticking ist spezialisiert auf die Herstellung von
Eishockeystöcken der Marke „Slapshot“.
In der Produktionsabteilung arbeiten 8 Personen, die jährlich 24.000
Stöcke produzieren.
An Roh-und Hilfsstoffen werden im Jahr 200 m3 Holz und 500 kg Leim
benötigt.
Die Einkaufspreise für 1 m3 Holz betragen 100 € und für 100 kg Leim 200 €.
Pro Jahr arbeitet ein Mitarbeiter durchschnittlich 1800 Stunden und
verdient 25 € in der Stunde.
Die Jahresrechnung für Energie beläuft sich auf 15.000 €, der
Jahresverbrauch auf 0,5 Mio. Kilowattstunden.
Ein Hockeystock kann für 22 € an den Fachhandel verkauft werden.

\begin{enumerate}[label=(\alph*)]
  \item Berechnen Sie die Produktivität und Wirtschaftlichkeit der Produktionsfaktoren Holz, Energie und Arbeit.
  \item Wie hoch ist die Gesamtkapitalrentabilität, wenn die übrigen jährlichen Betriebskosten 68.000 € betragen? Gehen Sie davon aus, dass erst am Ende der Periode Umsätze generiert werden.
  \item Wie verändern sich bei konstanter Produktionsmenge Produktivität und Wirtschaftlichkeit der einzelnen Produktionsfaktoren, wenn ein Mitarbeiter wegen seiner Pensionierung nur noch halbtags arbeitet, die Löhne um 4 \% angehoben werden, die Energiekosten bei gleichbleibendem Verbrauch auf 20.000 € steigen und der Verkaufspreis des einzelnen Stocks auf 24 € erhöht wird?
  \item Wie verändert sich die Rentabilität unter den in c) genannten Bedingungen?
\end{enumerate}

\solution{

\textbf{a) Produktivität und Wirtschaftlichkeit:}

\begin{itemize}
    \item \textbf{Produktivität} (Holz):
    \[
    Produktivität = \frac{Menge Output}{Menge Input} = \frac{24.000}{200} = 120 \, St\"ucke/m^3.
    \]

    \item \textbf{Wirtschaftlichkeit} (Holz):
    \[
    Wirtschaftlichkeit = \frac{Wert Output}{Wert Input} = \frac{24.000 \cdot 22}{200 \cdot 100} = \frac{528.000}{20.000} = 26,4.
    \]

    \item \textbf{Produktivität} (Energie):
    \[
    Produktivität = \frac{Menge Output}{Menge Input} = \frac{24.000}{500.000} = 0,048 \, St\"ucke/kWh.
    \]

    \item \textbf{Wirtschaftlichkeit} (Energie):
    \[
    Wirtschaftlichkeit = \frac{Wert Output}{Wert Input} = \frac{24.000 \cdot 22}{15.000} = \frac{528.000}{15.000} = 35,2.
    \]

    \item \textbf{Produktivität} (Arbeit):
    \[
    Produktivität = \frac{Menge Output}{Menge Input} = \frac{24.000}{8 \cdot 1.800} = \frac{24.000}{14.400} = 1,67 \, St\"ucke/Std.
    \]

    \item \textbf{Wirtschaftlichkeit} (Arbeit):
    \[
    Wirtschaftlichkeit = \frac{Wert Output}{Wert Input} = \frac{24.000 \cdot 22}{8 \cdot 1.800 \cdot 25} = \frac{528.000}{360.000} = 1,47.
    \]
\end{itemize}

\textbf{b) Gesamtkapitalrentabilität:}

Die Gesamtkosten setzen sich aus folgenden Bestandteilen zusammen:
\[
Gesamtkosten = Holzkosten + Leimkosten + Lohnkosten + Energiekosten + übrige Kosten.
\]
\[
Holzkosten = 200 \cdot 100 = 20.000 \, €, \quad Leimkosten = \frac{500}{100} \cdot 200 = 1.000 \, €.
\]
\[
Lohnkosten = 8 \cdot 1.800 \cdot 25 = 360.000 \, €, \quad Energiekosten = 15.000 \, €, \quad übrige Kosten = 68.000 \, €.
\]
\[
Gesamtkosten = 20.000 + 1.000 + 360.000 + 15.000 + 68.000 = 464.000 \, €.
\]

Der Gesamtumsatz beträgt:
\[
Umsatz = 24.000 \cdot 22 = 528.000 \, €.
\]

Der Gewinn berechnet sich wie folgt:
\[
Gewinn = Ertrag - Aufwand = 528.000 - 464.000 = 64.000 \, €.
\]

Die Rentabilität berechnet sich wie folgt:
\[
Rentabilität = \frac{Gewinn}{Eigenkapital} \cdot 100 = \frac{64.000}{464.000} \cdot 100 \approx 13,79 \, \%.
\]

\textbf{c) Änderungen bei konstanter Produktionsmenge:}
\begin{itemize}
    \item \textbf{Produktivität} (Holz):
    \[
    Alt: \quad \frac{24.000}{200} = 120 \, St\"ucke/m^3, \quad Neu: Unverändert.
    \]

    \item \textbf{Wirtschaftlichkeit} (Holz):
    \[
    Alt: \quad \frac{24.000 \cdot 22}{20.000} = 26,4, \quad Neu: \quad \frac{24.000 \cdot 24}{20.000} = 28,8.
    \]

    \item \textbf{Produktivität} (Arbeit):
    \[
    Alt: \quad \frac{24.000}{14.400} = 1,67 \, St\"ucke/Std., \quad Neu: \quad \frac{24.000}{13.500} \approx 1,78 \, St\"ucke/Std.
    \]

    \item \textbf{Wirtschaftlichkeit} (Arbeit):
    \[
    Alt: \quad \frac{24.000 \cdot 22}{360.000} = 1,47, \quad Neu: \quad \frac{24.000 \cdot 24}{351.000} \approx 1,64.
    \]

    \item \textbf{Wirtschaftlichkeit} (Energie):
    \[
    Alt: \quad \frac{24.000 \cdot 22}{15.000} = 35,2, \quad Neu: \quad \frac{24.000 \cdot 24}{20.000} = 28,8.
    \]
\end{itemize}

\textbf{d) Rentabilität unter neuen Bedingungen:}

Die neuen Gesamtkosten:
\[
Gesamtkosten = 20.000 + 1.000 + 351.000 + 20.000 + 68.000 = 460.000 \, €.
\]

Der neue Umsatz:
\[
Umsatz = 24.000 \cdot 24 = 576.000 \, €.
\]

Der Gewinn berechnet sich wie folgt:
\[
Gewinn = Ertrag - Aufwand = 576.000 - 460.000 = 116.000 \, €.
\]

Die Rentabilität berechnet sich wie folgt:
\[
Rentabilität = \frac{Gewinn}{Eigenkapital} \cdot 100 = \frac{116.000}{460.000} \cdot 100 \approx 25,22 \, \%.
\]
}

\exercise{1\textbar 4}{Unternehmensform}
Für Marlene Schuberth beginnt in drei Tagen das studienbegleitende Praxissemester. Sie studiert
Wirtschaftsinformatik im dritten Semester. In den ersten Semestern hat sie bereits einige Grundlagen und
Methoden der Betriebsführung in der Theorie kennengelernt. Nun ist sie ganz gespannt auf ihren beginnenden
Einsatz bei der OHM, einem Hersteller von Kaffeevollautomaten für den professionellen Bereich in Hotel und
Gastronomie.
\\~\\
Bevor sie ihren ersten Arbeitstag antritt, informiert sie sich noch einmal gründlich auf der Webseite des
Unternehmens. „OHM –Ihr Spezialist für den professionellen Kaffeegenuss“, mit diesem Slogan begrüßt sie die
Firmenpräsentation im Internet. Ihr Blick fällt auf den Menüpunkt „Über OHM“:
\\~\\
\textit{
Die 1890 gegründete OHM mit Sitz in Nürnberg gehört zu den führenden Herstellern vollautomatischer
Kaffeemaschinen für den Gastro-, Office-und Foodservicebereich. Kunden auf der ganzen Welt schätzen die
einfache Handhabung, eine große Getränkeauswahl auf Knopfdruck und die hervorragende Kaffeequalität. OHM
vereint Kaffeemaschinentechnologie und langjährige Kaffeekompetenz mit der Handwerkskunst eines Baristas.
Die Vollautomaten von OHM bereiten Getränke zu, die den handgemachten Kreationen eines Baristas in
Geschmack und Optik ebenbürtig sind.
}
\\~\\
Mit Vertriebsniederlassungen in Italien, Frankreich und der Schweiz sind die Produkte der OHM international
präsent.
\\~\\
Weitere Informationen erhält sie über das Impressum:
\\~\\
\textit{
OHM Elektrogeräte\\
Postanschrift:\\
Postfach 91 23 45, 90123 Nürnberg\\
Besucheranschrift:\\
Erlanger Straße 15, 90456 Nürnberg\\
Geschäftsführer: Otto Heinrich Meier\\
HRB 1234 Amtsgericht Nürnberg\\
Ust-IdNr.: DE 123459911
}
\\~\\
Beim weiteren Durchstöbern der Firmeninformationen stößt Marlene auch auf die Kennzahlen der
Geschäftsentwicklung. Die OHM hat im letzten Jahr einen Umsatz von 40 Mio. € erwirtschaftet.
In Deutschland sind ca. 250 Mitarbeiterinnen und Mitarbeiter bei OHM beschäftigt.
\\~\\
Diskutieren Sie die \textbf{unternehmerische Ausrichtung nach Standort, Branche, Rechtsform, Größe}.
Welche Güter erstellt die OHM?

\solution{

\begin{itemize}
    \item \textbf{Standort:} Der Hauptsitz ist in Nürnberg, Deutschland, mit internationalen Vertriebsniederlassungen in Italien, Frankreich und der Schweiz. Daher ist OHM \textbf{international} tätig.

    \item \textbf{Branche:} OHM gehört zum \textbf{\emph{sekundären Sektor}}, da das Unternehmen Güter (Kaffeevollautomaten) produziert.

    \item \textbf{Rechtsform:} Laut Impressum handelt es sich um eine \textbf{GmbH} (Gesellschaft mit beschränkter Haftung), da die Firma unter einer HRB-Nummer registriert ist.

    \item \textbf{Größe:} OHM beschäftigt ca. 250 Mitarbeiter und hat einen Jahresumsatz von 40 Mio. €. Laut EU-Definition gehört das Unternehmen zur Kategorie der \textbf{\emph{mittleren Unternehmen}} (Grenze: 50 Mio. € Umsatz und 250 Mitarbeiter). Eine genaue Einordnung als mittleres oder großes Unternehmen ist ohne weitere Bilanzinformationen schwierig.

    \item \textbf{Güter:} OHM produziert \textbf{vollautomatische Kaffeemaschinen} für den professionellen Einsatz in Gastronomie, Hotels und Büros.
\end{itemize}
}

\exercise{1\textbar 5}{Verbreitung verschiedener Rechtsformen}
Die Verbreitung verschiedener Rechtsformen stellte sich vereinfacht nach Daten des Statistischen Bundesamtes im Jahr 2018 wie folgt dar:

\begin{table}[h!]
\centering
\begin{tabular}{l|c}
\textbf{Rechtsform} & \textbf{Anteil 2018 in \%} \\
\hline
Einzelunternehmen & 61,6 \\
Personengesellschaften (OHG, KG, \dots) & 11,3 \\
Kapitalgesellschaften (GmbH und AG) & 21,1 \\
Sonstige Rechtsformen & 5,9 \\
\end{tabular}
\end{table}
\noindent
Suchen Sie nach Gründen für die unterschiedlich starke Verbreitung der verschiedenen Rechtsformen.

\solution{

\begin{itemize}
    \item \textbf{Einzelunternehmen (61,6\%):}
    Einzelunternehmen sind die am weitesten verbreitete Rechtsform, da sie einfach und kostengünstig zu gründen sind. Es wird kein Mindestkapital benötigt, und der bürokratische Aufwand ist gering. Diese Rechtsform ist besonders bei kleinen Unternehmen und Selbstständigen beliebt.

    \item \textbf{Personengesellschaften (11,3\%):}
    Personengesellschaften wie OHG und KG bieten die Möglichkeit, Ressourcen und Kompetenzen von mehreren Gesellschaftern zu bündeln. Sie sind flexibler als Kapitalgesellschaften, aber erfordern ein hohes Maß an Vertrauen zwischen den Gesellschaftern, da diese persönlich haften. Dies begrenzt ihre Verbreitung.

    \item \textbf{Kapitalgesellschaften (21,1\%):}
    Kapitalgesellschaften wie GmbH und AG sind besonders bei größeren Unternehmen beliebt, da die Haftung auf das Gesellschaftsvermögen beschränkt ist. Allerdings sind die Gründungskosten höher, und es wird ein Mindestkapital benötigt (z. B. 25.000 €{} bei einer GmbH). Diese Vorteile machen Kapitalgesellschaften attraktiv für Unternehmen, die größere Investitionen tätigen oder expandieren möchten.

    \item \textbf{Sonstige Rechtsformen (5,9\%):}
    Unter sonstigen Rechtsformen fallen beispielsweise Genossenschaften und eingetragene Vereine. Diese werden häufig in speziellen Kontexten verwendet, wie bei gemeinschaftlichen Projekten oder Non-Profit-Organisationen. Ihre Verbreitung ist aufgrund der eingeschränkten Anwendungsbereiche gering.
\end{itemize}
}

\exercise{1\textbar 6}{Rechtsformwahl bei Unternehmensgründung}
Sie haben von Ihrer Großmutter 30.000 Euro zum bestandenen Bachelorabschluss in Wirtschaftsinformatik
als Vorschuss auf Ihr Erbe geschenkt bekommen.
\\~\\
Während des Studiums haben Sie eine neue Software entwickelt, die Unternehmen Verkäufe im mobilen
Internet erleichtert. Das Gewinnpotenzial dieser Software ist auch nach Ansicht von Gründungsexperten hoch, allerdings drohen
Ihnen erhebliche Schadensersatzforderungen von Kunden, sollten mit der Software Computerviren
eingeschleust werden, was man nie ausschließen kann.
\\~\\
Sie wollen sich nun mit dieser Software selbstständig machen. Welche Rechtsform wäre geeignet?

\solution{

Die Wahl der Rechtsform sollte sowohl das Gewinnpotenzial als auch das Risiko von Schadensersatzforderungen berücksichtigen. Eine geeignete Rechtsform wäre:

\begin{itemize}
    \item \textbf{GmbH (Gesellschaft mit beschränkter Haftung):}
    \begin{itemize}
        \item Die Haftung ist auf das Gesellschaftsvermögen beschränkt, sodass Sie nicht mit Ihrem privaten Vermögen haften. Dies ist besonders wichtig, da bei Schadensersatzforderungen hohe Risiken bestehen.
        \item Das erforderliche Mindestkapital von 25.000 Euro kann durch die Schenkung der Großmutter gedeckt werden (30.000 Euro).
        \item Die GmbH bietet eine professionelle und vertrauenswürdige Außenwirkung, was im B2B-Bereich vorteilhaft ist.
    \end{itemize}

    \item \textbf{Alternativ: UG (haftungsbeschränkt):}
    \begin{itemize}
        \item Wenn Sie das Kapital für die GmbH nicht vollständig aufbringen möchten, könnte eine UG (Unternehmergesellschaft) eine Option sein. Diese Form erfordert lediglich 1 Euro Stammkapital.
        \item Die Haftung ist ebenfalls auf das Gesellschaftsvermögen beschränkt.
        \item Langfristig kann die UG in eine GmbH umgewandelt werden, sobald genug Kapital angespart ist.
    \end{itemize}

    \item \textbf{Einzelunternehmen: Nicht geeignet.}
    \begin{itemize}
        \item Bei einem Einzelunternehmen haften Sie mit Ihrem gesamten Privatvermögen, was angesichts der Schadensersatzrisiken nicht empfehlenswert ist.
    \end{itemize}
\end{itemize}

\textbf{Fazit:} Eine GmbH ist die bevorzugte Rechtsform, da sie das Risiko von Schadensersatzforderungen auf das Gesellschaftsvermögen beschränkt und gleichzeitig professionelle Außenwirkung bietet. Alternativ könnte eine UG als Einstieg sinnvoll sein, falls das Kapital für eine GmbH nicht vollständig eingebracht werden soll.
}

\exercise{1\textbar 7}{Analyse der Unternehmensumwelt}
Oft hört man die Feststellung, dass in der heutigen Zeit eine umfassende Umweltanalyse immer wichtiger für ein Unternehmen werde.

\begin{enumerate}[label=\alph*)]
    \item Welche Gründe können zu dieser Aussage führen?
    \item Skizzieren Sie eine Umweltanalyse für ein Fast-Food-Unternehmen wie McDonald’s. Gehen Sie dabei von den verschiedenen Umweltbereichen aus, denen sich ein Unternehmen gegenübersieht.
    \item Welche Maßnahmen schlagen Sie aufgrund der in b) gemachten Umweltanalyse für McDonald’s vor?
\end{enumerate}

\solution{

\begin{enumerate}[label=(\alph*)]
    \item \textbf{Gründe für die zunehmende Bedeutung der Umweltanalyse:}
    \begin{itemize}
        \item \textbf{Globalisierung:} Unternehmen müssen sich in internationalen Märkten und Lieferketten behaupten und auf globale Wettbewerbsbedingungen reagieren.
        \item \textbf{Technologischer Fortschritt:} Neue Technologien verändern Märkte und schaffen sowohl Risiken als auch Chancen, die analysiert werden müssen.
        \item \textbf{Veränderte Erwartungen von Stakeholdern:} Kunden, Staat und Öffentlichkeit erwarten nachhaltiges Handeln, Transparenz und die Einhaltung gesetzlicher Vorgaben.
        \item \textbf{Arbeitsmarkt:} Der Fachkräftemangel und steigende Anforderungen an Mitarbeitende machen es notwendig, Personalstrategien anzupassen.
        \item \textbf{Klimawandel und Nachhaltigkeit:} Unternehmen müssen auf strengere Umweltauflagen reagieren und ihre Geschäftsmodelle an neue Bedingungen anpassen.
    \end{itemize}

    \item \textbf{Umweltanalyse für ein Fast-Food-Unternehmen wie McDonald’s:}
    \begin{itemize}
        \item \textbf{Politischer Bereich:}
        \begin{itemize}
            \item Welche Bau- und Umweltauflagen gelten für neue Standorte?
            \item Wie reagieren lokale Behörden auf die Eröffnung eines Fast-Food-Restaurants (z. B. Gewerbesteuern, Verkehr, Lärm)?
        \end{itemize}

        \item \textbf{Wirtschaftlicher Bereich:}
        \begin{itemize}
            \item Welche Lieferanten können für Fleisch, Brot und Gemüse genutzt werden (lokal vs. international)?
            \item Wie entwickeln sich Rohstoffpreise und Transportkosten?
            \item Wie sieht der Arbeitsmarkt aus (z. B. Verfügbarkeit von Studierenden als Teilzeitkräfte)?
        \end{itemize}

        \item \textbf{Sozialer Bereich:}
        \begin{itemize}
            \item Wie reagieren die Anwohner und die lokale Bevölkerung auf ein neues Restaurant?
            \item Welche Erwartungen haben Kunden an das Angebot (z. B. vegane Optionen, nachhaltige Verpackungen)?
        \end{itemize}

        \item \textbf{Technologischer Bereich:}
        \begin{itemize}
            \item Wie können Selbstbedienungskioske und digitale Bestellsysteme eingesetzt werden?
            \item Welche Innovationen in der Lebensmitteltechnologie könnten genutzt werden?
        \end{itemize}

        \item \textbf{Ökologischer Bereich:}
        \begin{itemize}
            \item Wie können Verpackungsmaterialien nachhaltiger gestaltet werden?
            \item Wie kann der CO₂-Ausstoß durch Transport und Produktion reduziert werden?
        \end{itemize}

        \item \textbf{Rechtlicher Bereich:}
        \begin{itemize}
            \item Welche Steuern und Abgaben müssen entrichtet werden?
            \item Welche Vorschriften gelten für die Müllentsorgung und Hygiene?
        \end{itemize}
    \end{itemize}

    \item \textbf{Maßnahmen auf Basis der Umweltanalyse:}
    \begin{itemize}
        \item \textbf{Nachhaltige Lieferkette:}
        Zusammenarbeit mit lokalen Lieferanten, um Transportwege zu verkürzen und regionale Wirtschaft zu stärken. Verwendung nachhaltiger und umweltfreundlicher Zutaten.

        \item \textbf{Technologieeinsatz:}
        Einführung digitaler Bestellsysteme zur Effizienzsteigerung und Senkung der Personalkosten. Investitionen in energieeffiziente Küchentechnologie.

        \item \textbf{Soziale Verantwortung:}
        Angebot gesünderer und nachhaltiger Speisen, wie vegane und vegetarische Optionen. Förderung der Akzeptanz durch Zusammenarbeit mit der lokalen Gemeinde.

        \item \textbf{Umweltschutz:}
        Verzicht auf Einwegplastik, Einführung von Recyclingprogrammen und Nutzung kompostierbarer Verpackungen.

        \item \textbf{Personalentwicklung:}
        Attraktive Arbeitsbedingungen schaffen, z. B. durch flexible Arbeitszeiten und Schulungsprogramme, um qualifizierte Mitarbeitende zu gewinnen und zu halten.
    \end{itemize}
\end{enumerate}
}

\exercise{1\textbar 8}{Standortentscheidung}
„Guten Tag, Frau Schuberth, herzlich willkommen! Wir sitzen hier gerade zusammen, um einen weiteren Lagerstandort zu finden. Wir planen den Bau einer neuen Fertigungsstraße. Da wir hier aus allen Nähten platzen, muss ein Teil unseres Lagers geopfert werden. Wir werden einige unserer produzierten Artikel auslagern müssen. Der Auftrag an unseren Immobilienmakler lautet: Die Nähe zum Produktionsstandort ist uns sehr wichtig, die Entfernung sollte unter 30 km liegen. Wir benötigen kurzfristig 1.000 qm, längerfristig denken wir an eine Erweiterung auf bis zu 3.000 qm. Eine Miete von 4 € pro qm wäre angemessen.“
\\~\\
Unsere Entscheidungskriterien haben wir folgendermaßen gewichtet:

\begin{table}[h!]
\centering
\begin{tabular}{l|c}
\textbf{Kriterium} & \textbf{Gewichtung (in \%)} \\
\hline
Entfernung zum Produktionsstandort & 30 \\
Lagergröße & 20 \\
Verfügbarkeit & 20 \\
Miethöhe & 10 \\
Verkehrsanbindung/Erreichbarkeit & 10 \\
Instandsetzungskosten & 10 \\
\end{tabular}
\end{table}
\noindent
Wegen der zu großen Entfernung wurden bereits Standorte in Bamberg und Forchheim ausgeschlossen. Eine Immobilie in Fürth, die eine Lagerfläche von nur 540 qm aufzuweisen hatte, wurde ebenfalls sofort aussortiert. Derzeit haben wir drei Immobilien in die engere Auswahl genommen. Nun müssen wir eine Entscheidung treffen.
\\~\\
\textbf{Dies sind die drei Angebote:}
\\~\\
\textbf{Fürth:}\\
In Fürth hätten wir eine Lagerhalle mit \textbf{2.480 qm} zu einer Miete von \textbf{3,80 € pro qm}. Die Miete entspricht unseren Erwartungen, ist jedoch für die Lage eher etwas über dem Durchschnitt. Die Größe ist ausreichend und hat auch noch Potential für weitere Planungen. Die Halle ist in Fürth/Poppenreuth mit unmittelbarer Anbindung an die \textbf{A73}. Dies ist für uns eine mittlere bis gute Entfernung zum Standort. Die verkehrstechnische Erreichbarkeit würde ich auch eher im Mittel sehen. An der Lagerhalle sind \textbf{keinerlei Renovierungsarbeiten nötig}, es wäre sogar bereits ein Hochregallager vorhanden. Allerdings ist diese Immobilie erst in \textbf{einem halben Jahr verfügbar}. Das wäre für unsere Planung etwas zu knapp.
\\~\\
\textbf{Nürnberg:}\\
Im \textbf{Nürnberger Hafen} haben wir eine Immobilie, die unserem Produktionsstandort \textbf{am nächsten liegt}. Verkehrstechnisch eine ausgezeichnete Lage direkt an der \textbf{A73} und kurze Verbindungen zur \textbf{A3}, \textbf{A9} und \textbf{A7}. Mit \textbf{9.050 qm} ist diese Immobilie eher zu groß. Die Immobilie kann aber geteilt werden zu etwa \textbf{4.500 qm}, was immer noch für unseren vorrangigen Nutzen zu groß wäre. Mit \textbf{4,80 € pro qm} ist diese Immobilie unter all den Angeboten \textbf{am teuersten}. Die Lagerhalle wäre \textbf{sofort verfügbar}. Eine Paletten-Regalanlage muss noch eingebaut werden, das ist für uns ein \textbf{geringer bis mittlerer Aufwand zur Instandsetzung}.
\\~\\
\textbf{Erlangen:}\\
In Erlangen hätten wir eine Immobilie mit einer \textbf{idealen Lagergröße von 1.300 qm}. Mit nur \textbf{2 € pro qm} ein \textbf{Spitzenangebot}. Allerdings kommen hier \textbf{hohe Renovierungskosten} auf uns zu. Die Halle ist \textbf{sehr alt} und müsste vollständig überholt werden. Die Immobilie wäre zwar \textbf{sofort verfügbar}, aber durch die notwendigen Renovierungsarbeiten würde sich der Start der Nutzung noch etwas aufschieben. Diese Lagerhalle ist unter den drei Angeboten \textbf{am weitesten von unserem Produktionsstandort entfernt}. Die \textbf{A3} ist hier in nächster Nähe, doch wir alle wissen, dass es an dieser Stelle immer wieder zu \textbf{Staus und Verkehrsengpässen} kommt.
\\~\\
Aufgaben
\begin{enumerate}[label=(\alph*)]
  \item Erstellen Sie eine Nutzwertanalyse für die OHM GmbH. Verwenden Sie eine Bewertungsskala von 1 bis 10, wobei die Höchstwertung 10 den maximalen Erfüllungsgrad bedeutet.
  \item Diskutieren Sie Vor- und Nachteile einer Nutzwertanalyse.
\end{enumerate}

\solution{

\begin{enumerate}[label=(\alph*)]
    \item \textbf{Nutzwertanalyse:}

    
\begin{table}[H]
\centering
\begin{tabular}{l|c|cc|cc|cc}
\textbf{Kriterien} & \textbf{Gewichtung} & \multicolumn{2}{c|}{\textbf{Fürth}} & \multicolumn{2}{c|}{\textbf{Nürnberg}} & \multicolumn{2}{c}{\textbf{Erlangen}} \\
\hline
 &  & Bewertung & Ergebnis & Bewertung & Ergebnis & Bewertung & Ergebnis \\
\hline
Entfernung & 30 & 8 & 2,4 & 10 & 3,0 & 1 & 0,3 \\
Größe & 20 & 10 & 2,0 & 2 & 0,4 & 10 & 2,0 \\
Verfügbarkeit & 20 & 3 & 0,6 & 10 & 2,0 & 8 & 1,6 \\
Miethöhe & 10 & 9 & 0,9 & 2 & 0,2 & 10 & 1,0 \\
Verkehrsanbindung & 10 & 6 & 0,6 & 10 & 1,0 & 4 & 0,4 \\
Instandsetzung & 10 & 10 & 1,0 & 7 & 0,7 & 1 & 0,1 \\
\hline
\textbf{Summe} & \textbf{100} & \textbf{} & \textbf{7,5} & \textbf{} & \textbf{7,3} & \textbf{} & \textbf{5,7} \\
\end{tabular}
\end{table}

\noindent
    \item \textbf{Vor- und Nachteile einer Nutzwertanalyse:}
    \textbf{Vorteile:}
    \begin{itemize}
        \item Entscheidungen können transparent gefällt werden.
        \item Die Entscheidungsfindung liegt schriftlich vor und kann auch in der Zukunft nachvollzogen werden.
        \item Die Nutzwertanalyse kann gut im Team und/oder von verschiedenen Personen durchgeführt werden und als Diskussionsgrundlage dienen.
    \end{itemize}
    \textbf{Nachteile:}
    \begin{itemize}
        \item Die Bewertung ist recht subjektiv. Die Festlegung der Gewichtungen und die Vergabe von Punkten sind keine exakt messbaren Vorgänge.
        \item Bei sehr vielen Alternativen und/oder Bewertungskriterien wird die Methode schnell zeitaufwändig.
    \end{itemize}

\end{enumerate}
}

\exercise{1\textbar 9}{Standortfaktoren}
Bei der Wahl eines Unternehmensstandortes werden meistens die wesentlichen Standortfaktoren zusammengestellt.

\begin{enumerate}[label=(\alph*)]
    \item Was versteht man unter Standortfaktoren?
    \item Welche Probleme stellen sich bei der Zusammenstellung dieser Standortfaktoren?
\end{enumerate}

\solution{

\begin{enumerate}[label=(\alph*)]
    \item \textbf{Was versteht man unter Standortfaktoren?}

    Faktoren, die die Wahl eines Standortes maßgeblich beeinflussen, werden als Standortfaktoren bezeichnet. Von Bedeutung können vor allem folgende Standortfaktoren sein:
    \begin{itemize}
        \item \textbf{Arbeitsbezogene Standortfaktoren:} Kosten und Qualifikation der Arbeitskräfte.
        \item \textbf{Materialbezogene Standortfaktoren:} Transportkosten, Zuliefersicherheit, Art der zu beschaffenden Inputfaktoren (z. B. Rohstoffe), Verfügbarkeit und Preise von Energie.
        \item \textbf{Absatzbezogene Standortfaktoren:} Kundennähe, potenzielle Nachfrage und Konkurrenzsituation.
        \item \textbf{Verkehrsbezogene Standortfaktoren:} Verkehrsinfrastruktur, die geringe Transportkosten und -zeit verursacht.
        \item \textbf{Immobilienbezogene Standortfaktoren:} Mietpreisniveau, Immobilienpreise.
        \item \textbf{Umweltbezogene Standortfaktoren:} Besondere Auflagen, Schutzgebiete.
        \item \textbf{Abgabenbezogene Standortfaktoren:} Besteuerung.
        \item \textbf{Clusterbildung:} Vorhandensein von Know-how-Trägern in einer Region.
    \end{itemize}
    Zusätzlich können genannt werden:
    \begin{itemize}
        \item Rechtliche und politische Stabilität.
    \end{itemize}

    \item \textbf{Welche Probleme stellen sich bei der Zusammenstellung dieser Standortfaktoren?}

    \begin{itemize}
        \item \textbf{Skala der Bewertung:} Wie kann nach den verschiedenen Faktoren und Kriterien differenziert werden?
        \item \textbf{Gewichtung:} Wie werden die relevanten Faktoren untereinander gewichtet? Gibt es Mindestanforderungen an einzelne Standortfaktoren, die in jedem Fall erfüllt sein müssen?
        \item \textbf{Quantifizierbarkeit der Standortfaktoren:} Viele Faktoren lassen sich überhaupt nicht oder nur schwer quantifizieren.
        \item \textbf{Vergleich:} Wie lässt sich der gleiche Standortfaktor bei einem Vergleich unter verschiedenen Standorten bewerten und gewichten? Beispielsweise haben Parkplätze am Bahnhof in München ein kleineres Gewicht als in einem Einkaufszentrum auf dem Lande.
        \item \textbf{Prognose:} Können die relevanten Standortfaktoren überhaupt in der gewünschten Genauigkeit erfasst werden?
        \item \textbf{Objektivität:} Die Bewertung und Gewichtung hängt von den persönlichen Wertvorstellungen der Anwender ab.
        \item \textbf{Fristigkeit:} Die Standortfaktoren sind im Zeitablauf – vor allem in mittel- und langfristiger Sicht – Veränderungen unterworfen, sodass sich auch ihre Gewichtung und Bewertung verschiebt.
    \end{itemize}
\end{enumerate}
}

\exercise{1\textbar 10}{Richtig oder falsch?}
Im Folgenden finden Sie fünf Aussagen. Bitte begründen Sie, ob diese stimmen, zum Teil beziehungsweise unter bestimmten Bedingungen stimmen oder ob die Aussagen falsch sind!\\
\note{\textbf{Tipp:} Die Aufgaben sind nicht leicht, antworten Sie nicht vorschnell, sondern lesen Sie die Aufgaben genau durch und denken Sie lieber zweimal nach!}

\begin{enumerate}
    \item Wirtschaften von Unternehmen in einer Marktwirtschaft dient der Gewinnerzielung und nicht der Bedürfnisbefriedigung!
    \item Unternehmen müssen auf ihre Gewinne Umsatzsteuer zahlen!
    \item Wenn die Aktienkurse einer Aktiengesellschaft (AG) steigen, dann hat das Unternehmen mehr liquide Mittel zur Verfügung!
    \item Der zunehmende Wettbewerb durch das Internet sorgt dafür, dass immer mehr Unternehmen im stationären Geschäft Konkurrenzmeider sind!
    \item In Dienstleistungsunternehmen ist die Wertschöpfungskette verkürzt (keine Logistik, wenig Einkauf). Deshalb ist die Wertschöpfungstiefe in der Regel geringer als bei Produktionsbetrieben (etwa Automobilherstellern)!
\end{enumerate}

\solution{

\begin{enumerate}
    \item \textbf{Falsch.} \\
    Wirtschaften von Unternehmen dient in einer Marktwirtschaft zwar aus Sicht der Unternehmen in erster Linie der Gewinnerzielung. Allerdings erzielen diese nur Gewinne, wenn damit auch Bedürfnisse befriedigt werden und Kunden bereit sind, Geld für die erstellten Produkte oder Dienstleistungen zu bezahlen.\\
    Über den Marktmechanismus dient das Wirtschaften von Unternehmen also auch der Bedürfnisbefriedigung.

    \item \textbf{Falsch.} \\
    Unternehmen zahlen auf Gewinne keine Umsatzsteuer, sondern Gewinnsteuer. Diese erfolgt als Einkommensteuer (bei Personengesellschaften) oder Körperschaftsteuer (bei Kapitalgesellschaften).\\
    Umsatzsteuer, auch Mehrwertsteuer genannt, wird auf den erzielten Umsatz bezahlt.

    \item \textbf{Falsch.} \\
    Der Aktienkurs einer bereits am Markt befindlichen Aktie hat für die Finanzlage des Unternehmens direkt keine Bedeutung. Unternehmen erhalten nur bei Erstausgabe einer Aktie Geld und zwar in Höhe des Ausgabekurses.\\
    Spätere Verkäufe haben auf die Vermögens- und Finanzanlage des Unternehmens direkt keinen Einfluss. Hier verkaufen lediglich Aktionäre ihre Eigentumsanteile an andere Aktionäre weiter und erzielen dabei persönliche Gewinne oder Verluste.

    \item \textbf{Falsch.} \\
    Seltsamerweise ist eher das Gegenteil der Fall. Durch E-Commerce, also Verkäufe über das Internet, hat sich die Wettbewerbssituation tatsächlich verschärft.\\
    Möchten Handelsbetriebe, die stationäre Geschäfte betreiben, mithalten, so suchen sie nun eher Wettbewerbsvorteile, weil dem Kunden durch Ballung vieler Geschäfte zumindest ein größeres Einkaufserlebnis geboten wird und er auch schnell verschiedene Angebote vergleichen kann, so wie im Internet eben.

    \item \textbf{Falsch.} \\
    In Dienstleistungsunternehmen ist die Wertschöpfungstiefe oftmals sogar größer als in Produktionsbetrieben, da man zur Dienstleistung keine Vorprodukte erwerben kann.\\
    Automobilunternehmen setzen zunehmend nur noch auf Endmontage und lassen sich fertige Module (Stoßfängersysteme, Getriebe, Sitze usw.) von Zulieferern anliefern.\\
    Dadurch ist die Wertschöpfungstiefe gering.\\
    Wenn Sie zum Friseur gehen, der ein klassischer Dienstleister ist, so kann der Friseur nicht einen halb fertigen Haarschnitt von einem Lieferanten zukaufen und dann nur noch den Rest erledigen.\\
    Deswegen ist hier die Wertschöpfungstiefe relativ hoch.
\end{enumerate}
}
