\exercise{4|1}{SWOT-Analyse}{
Der Umsatz der OHM GmbH soll in den nächsten 5 Jahren gesteigert werden. Es gibt Rücklagen in Höhe von 2 Mio. €, die in die Weiterentwicklung des Unternehmens investiert werden sollen.
\\~\\
\textit{„Darf ich vorstellen, das ist unsere Praktikantin Marlene Schuberth. Sie unterstützt uns diese Woche im Marketing.“,} teilt der Praktikantenbetreuer mit. Dieser fährt fort: \textit{„Frau Schuberth, ich bin sehr stolz darauf, dass wir generell keine so große Fluktuation unter den Mitarbeiterinnen und Mitarbeitern zu verzeichnen haben. Das ist für mich ein Zeichen, dass sich alle sehr wohl fühlen. In nächster Zeit wird allerdings ein ganzer Schwung in den Ruhestand wechseln. Wir werden also einige Neueinstellungen vor uns haben, um die Lücken wieder zu schließen.“}
\\~\\
Es übernimmt die Leiterin der Marketing-Abteilung: \textit{„Liebe Kollegen, wir könnten nun mit meiner Voranalyse starten, doch da Frau Schuberth noch ganz neu bei uns ist, möchte ich diese Runde mit einer Betrachtung der Ist-Situation beginnen. Frau Schreiber, schildern Sie doch bitte kurz, wie Sie derzeitig unsere Marktsituation sehen.“}
\\~\\
Die Dame in der Runde beginnt zu erzählen. Marlene schätzt Frau Schreiber auf Mitte 50, als Vertrieblerin scheint sie bereits viele Erfahrungen gesammelt zu haben. \textit{„Unsere Kunden schätzen die hohe Qualität unserer Maschinen. Wir sind im Hotel- und Gastronomiebereich mit einer der führenden Hersteller. Wir bieten für jeden Anspruch die ideale Kaffeemaschine. Angefangen bei unserer ,Kleinen‘, der Coffee Easy, bis hin zu unserer modular ausbaubaren Coffee Vario. Mit unseren zahlreichen Beistellgeräten können wir jedem Kunden eine individuelle Maschinenkonfiguration anbieten. Unsere Maschinen produzieren alle einen Kaffee von hervorragender Qualität. Das wissen unsere Kunden und viele vertrauen uns schon seit Jahren. Wir betreiben ein eng geknüpftes Service- und Wartungsnetzwerk und legen großen Wert auf die technische Kompetenz unseres Außendienstes. Alles in allem bieten wir einen ausgezeichneten technischen Support während der gesamten Lebensdauer unserer Maschinen.“}
\\~\\
Ein weiterer Vertriebler sagt: \textit{„Wir stehen immer in engem Kontakt mit unserem Außendienst. Deren Erfahrungen beim Kunden vor Ort laufen direkt wieder in unsere Entwicklung ein. Wir können auf eine langjährige Erfahrung in der Produktion von Vollautomaten bauen. Unsere Kunden schätzen die Präzision und Zuverlässigkeit unserer Automaten. Wir sind sehr stolz auf einige unserer Produktinnovationen. Besonders für die Milchverarbeitung haben wir ausgereifte Technologien entwickelt.“}
\\~\\
Frau Schreiber führt fort: \textit{„Hinter unseren Innovationen steckt natürlich unser umfassendes Wissen über die Verarbeitung und Zubereitung von Kaffee. Dieses Know-how geben wir gerne direkt an unsere Kunden weiter. In unserem Schulungscenter in Nürnberg bieten wir neben unseren Produktschulungen auch Kurse zur Zubereitung von Kaffee und Latte-Getränken. Als Kursleiter konnten wir prämierte Barista gewinnen. Unsere sämtlichen Techniker werden hier ebenso geschult und weitergebildet. Zusammenfassend kann man also sagen: Wir haben qualitativ hochwertige und auch hochpreisige Kaffeevollautomaten, die wir für unsere Kunden individuell anpassen können. Wir legen großen Wert auf unsere zusätzlichen Serviceleistungen. Unsere Kunden schätzen uns und wir haben eine gute Marktposition im Hotel- und Gastronomiebereich. Allerdings sehen wir, gerade auf dem deutschen Markt, nur noch ein geringes Wachstum. Seit einigen Jahren beobachten wir die Latte Macchiatisierung der deutschen Haushalte. Wo früher eine Filterkaffeemaschine stand, steht heute der technisch ausgefeilte Vollautomat. Die Verbraucher haben heute einfach keine Angst mehr vor zu viel Technik.“}
\\~\\
Ein Assistent drückt kurz eine Taste seines geöffneten Notebooks, und folgende Grafik wird an die Wand geworfen: 
\\~\\
\textit{„In den deutschen Haushalten finden Sie überwiegend noch die gute alte Brühmaschine. Der Anteil der klassischen Filterkaffeemaschine ist in den letzten Jahren allerdings stetig gefallen und stagniert nun. Gewinner sind nach wie vor die Kaffeevollautomaten und die Kapselmaschinen, die einen stetigen Aufwärtstrend fortsetzen. Die Vollautomaten und Kapselmaschinen nehmen heute sogar den Pad-Maschinen Marktanteile ab. Soweit die Beobachtungen auf dem Haushaltsmarkt. Wir stellen derzeit fest, dass immer mehr Haushaltsmaschinen in den Hotels ihre Verwendung finden. In Besprechungszimmern, Aufenthaltsräumen, Hotellobbys stehen diese kleinen und peppig designten Maschinen. Der Hotelgast kennt die Bedienung bereits von Zuhause und hat keinerlei Scheu davor. Bisher haben wir uns ganz bewusst von dem Markt für Haushaltskaffeemaschinen distanziert und haben daher bei den Verbrauchern nur eine geringe Bekanntheit. Aufgrund unserer Beobachtungen dürfen wir diese Zielgruppe in Zukunft nicht unbeachtet lassen. Deswegen sind wir heute zusammengekommen, um Ideen zu sammeln und zukünftige Investitionen in die richtige Richtung zu lenken.“}
\\~\\
Frau Schreiber ergreift energisch das Wort: \textit{„Wir haben doch keinerlei Erfahrung mit den Haushalten. Unsere Produkte sind dort nicht etabliert. Die Kaufentscheidungen laufen dort ganz anders ab, unser Vertrieb ist darauf gar nicht eingerichtet.“} Herr Zwingel stimmt ihr völlig zu: \textit{„Es gibt doch viel zu viele Mitbewerber im Haushaltssektor, wie wollen wir uns dort absetzen? Unsere Automaten sind hochpreisige Maschinen, die mit diesen Billig-Geräten gar nicht zu vergleichen sind.“}
\\~\\
Herr Meier unterbricht die aufgebrachten Kollegen: \textit{„Ihre Einwände sind alle ganz richtig. Wir müssen aber doch Trends und Marktveränderungen um uns herum beobachten und rechtzeitig darauf reagieren. Unsere Mitbewerber dürfen nicht schneller mit neuen Ideen am Markt sein. Wir wollen mit unserem innovativen Team weiterhin führend sein. Das alles wird mit Sicherheit große Veränderungen hervorbringen, vielleicht auch mit einer Auswirkung auf das Personal.“}
\\~\\
Frau Schreiber meldet sich zu Wort: \textit{„Ich finde es schön und auch wichtig, sich nach Trends auszurichten. Wir sehen seit einigen Jahren auch die Trends zu Bio-Produkten, die Nachfrage nach Nachhaltigkeit und Ethik in der Kaffeeproduktion. Auch diese Trends sollten wir als Möglichkeiten bei unseren Überlegungen mit einschließen.“}
\\~\\
Herr Lauer setzt fort: \textit{„Daher habe ich den Kapselmarkt bereits etwas näher betrachtet. Die Maschinen werden im Niedrigpreis-Segment angeboten, gepaart mit einer Kundenbindung über eine exklusive Clubmitgliedschaft oder Gutschriften über Kaffeekapseln. Die Einnahmen laufen hier über den Verbrauch der Kapseln. Das ist eine völlig andere Preisstrategie. Andererseits haben wir bereits gute Kontakte zu verschiedenen Röstereien. Ich sehe hier eine sehr gute Möglichkeit, eine Zusammenarbeit oder Partnerschaft mit einer ausgewählten Rösterei einzugehen. Auf dem Kapselmarkt existieren allerdings viel zu viele Mitstreiter. Wir sollten etwas Neues und Innovatives auf den Markt bringen. Ich setze da eher auf unser Entwicklerteam.“}
\\~\\
Der Entwickler meldet sich zu Wort: \textit{„Im Consumer-Bereich sehe ich viele neue technischen Möglichkeiten. Einige meiner Kolleginnen und Kollegen haben bereits mit einer App-Entwicklung die Bedienung übers Smartphone gestartet. Allerdings haben wir dies für unsere Businesskunden nie in Betracht gezogen. Internet of Things (IoT) wird für die Haushalte der Zukunft immens wichtig. Auch dazu haben wir bereits viele Ideen.“}
\\~\\
Herr Meier springt erfreut auf: \textit{„Das ist toll, richten Sie mit Ihrem Team all ihre Ideen auf unsere kleinste Maschine, die Coffee Easy. Der Fokus muss dabei aber auf Design und Emotion liegen. Der Verbraucher trifft seine Kaufentscheidung viel schneller und vor allem emotionaler als unsere bisherigen Businesskunden. Damit haben wir wenig Erfahrung, aber ich finde das spannend.“}
\\~\\
Frau Schreiber bemerkt etwas kritisch: \textit{„Wir platzen hier am Standort Nürnberg aus allen Nähten, wo soll denn die Produktion einer neuen Produktlinie stattfinden?“} \textit{„Ja, Frau Schreiber, das ist nur eine von vielen offenen Fragen, die wir in Zukunft zu klären haben. Ich werde mich auch mit unseren Personalern zusammensetzen müssen, um der Überalterung in unserem Betrieb entgegenzugehen. Gleichzeitig müssen die Finanzen für unser Vorhaben geschätzt und die Ausgaben geplant werden. Es gibt also viel zu tun bis zur nächsten Besprechung.“}
\\~\\
Erstellen Sie ein Stärken-Schwächen-Profil (SWOT-Analyse).

\solution{

\begin{table}[h!]
\centering
\begin{tabular}{|p{7.5cm}|p{7.5cm}|}
\hline
\textbf{Stärken} & \textbf{Schwächen} \\ \hline
\begin{itemize}
    \item Hochwertige und hochpreisige Produkte, hohe Qualität
    \item Etabliert und führend im Hotel- und Gastronomiebereich
    \item Vollautomaten für jeden Anspruch, individuell und modular anpassbar
    \item Hervorragende Kaffeequalität
    \item Starkes Kundenvertrauen
    \item Langjährige Erfahrung, Know-how
    \item Personal: langjährige und engagierte Mitarbeiter mit großer Erfahrung
    \item Know-how-Transfer im Schulungscenter
    \item Serviceorientiert, guter Kundenservice und Wartungsnetzwerk
    \item Zuverlässige und innovative Technik
\end{itemize} & 
\begin{itemize}
    \item Personal: Überalterung, Nachwuchsprobleme
    \item Geringer Wachstum im Gastbereich
    \item Geringe Bekanntheit, wenig ausgeprägte Corporate Identity
    \item Fehlende Erfahrung mit der Zielgruppe Consumer
    \item Vertrieb ist nicht auf Consumer ausgerichtet
    \item Nicht im Haushaltsbereich etabliert
    \item Geräte zu teuer, hoher Anschaffungspreis
    \item Standort: Raumnot
    \item Finanzierung ungeklärt
\end{itemize} \\ \hline

\textbf{Chancen} & \textbf{Risiken} \\ \hline
\begin{itemize}
    \item Gewinnen einer neuen Zielgruppe
    \item Trend zu Bio-Produkten
    \item Nachfrage nach Nachhaltigkeit und Ethik in der Kaffeeproduktion
    \item Zusammenarbeit mit Rösterei
    \item Verfolgen von technischen Trends
    \item App-Anbindung, IoT-Integration
\end{itemize} &
\begin{itemize}
    \item Haushaltsmaschinen rücken in den Gastbereich vor
    \item Viele Wettbewerber im Haushaltsbereich
    \item Wettbewerber bieten billigere Produkte an
    \item Wettbewerber sind schneller, technische Trends werden verpasst
\end{itemize} \\ \hline
\end{tabular}
\end{table}
}

\exercise{4|2}{Marktanteils- und Marktwachstumsberechnung}{
Die Meier \& Partner GmbH fertigt am Standort München Fräsmaschinen für die metallverarbeitende Industrie.
Im Geschäftsjahr 2020 konnte das Unternehmen 2.000 Maschinen im deutschen Markt absetzen.
Der Branchenverband beziffert das gesamte Marktvolumen in 2020 auf 12.000 Einheiten.
Der größte nationale Konkurrent und Marktführer, die Fräs und Schleif AG mit Sitz in Duisburg, nennt im Jahresbericht ein
Absatzvolumen von 4.000.

Aufgrund der günstigen Konjunktur wird ein Anstieg des Marktvolumens im Jahr 2021 auf 14.000 Maschinen erwartet.

Berechnen Sie:
\begin{enumerate}[label=\alph*)]
    \item den absoluten Marktanteil der Meier \& Partner GmbH in 2020
    \item den absoluten Marktanteil der Fräs und Schleif AG in 2020
    \item den relativen Marktanteil der Meier \& Partner GmbH in 2020
    \item das Marktwachstum für das Jahr 2021
\end{enumerate}
}

\solution{

Absoluter Marktanteil der Meier \& Partner GmbH in 2020:
    \[
    Absoluter Marktanteil = \frac{Absatz des Unternehmens}{Marktvolumen} \cdot 100
    \]
    \[
    Absoluter Marktanteil Meier \& Partner GmbH = \frac{2.000}{12.000} \cdot 100 = 16,67\,\%
    \]

Absoluter Marktanteil der Fräs und Schleif AG in 2020:
    \[
    Absoluter Marktanteil Fräs und Schleif AG = \frac{4.000}{12.000} \cdot 100 = 33,33\,\%
    \]

Relativer Marktanteil der Meier \& Partner GmbH in 2020:
    \[
    Relativer Marktanteil = \frac{Marktanteil des Unternehmens}{Marktanteil des stärksten Konkurrenten} \cdot 100
    \]
    \[
    Relativer Marktanteil Meier \& Partner GmbH = \frac{16,67}{33,33} \cdot 100 = 50\,\%
    \]

Marktwachstum für das Jahr 2021:
    \[
    Marktwachstum = \frac{Erwartetes Marktvolumen 2021 - Marktvolumen 2020}{Marktvolumen 2020} \cdot 100
    \]
    \[
    Marktwachstum = \frac{14.000 - 12.000}{12.000} \cdot 100 = 16,67\,\%
    \]
}
\exercise{4|3}{Preisstrategien}

Welche unterschiedlichen Preisstrategien können Sie im Produktlebenszyklus verfolgen? \\
Welche Strategie empfehlen Sie jeweils für:
\begin{enumerate}[label=\alph*)]
    \item die neue Sony-Playstation-Spielekonsole,
    \item das neue Modell Sandero der Automarke Dacia und
    \item die neue Sonnenbrillenkollektion von Puma?
\end{enumerate}

\solution{
\begin{itemize}
    \item \textbf{Abschöpfungsstrategie:} \\
    Bei der Abschöpfungsstrategie gehen Sie mit einem hohen Preis in den Markt und lassen diesen im Lauf der Zeit immer weiter sinken. 
    Das geht vor allem dann, wenn Ihr neues Produkt einen Konkurrenzvorsprung hat, der nicht einfach kopierbar ist. 
    Das wäre eine passende Strategie für die neue \textit{Sony Playstation}, die bei Markteinführung den Konkurrenzprodukten technisch vermutlich überlegen ist.
    
    \item \textbf{Markenpreisstrategie:} \\
    Eine Markenpreisstrategie können Sie sich erlauben, wenn Sie eine Marke haben, die von den Käufern als wertvoll empfunden wird und für die Ihre Klientel bereitwillig mehr zahlt. 
    Sie gehen dann mit einem Preis über dem Marktdurchschnitt auf den Markt und lassen den Preis dauerhaft hoch. 
    Diese Strategie würde sich für die \textit{Puma-Sonnenbrillenkollektion} empfehlen.
    
    \item \textbf{Durchdringungsstrategie:} \\
    Die Durchdringungsstrategie zielt darauf, dass Sie am Markt schnell einen hohen Marktanteil erzielen, daher hohe Mengen produzieren können und dadurch kostengünstiger produzieren als die Konkurrenz. 
    So verschaffen Sie sich einen strategischen Wettbewerbsvorteil. 
    Um Marktanteile zu erobern, gehen Sie mit einem niedrigen Preis in den Markt und lassen ihn auch vorläufig niedrig. 
    Das ist die Preisstrategie, die \textit{Dacia} bisher verfolgt und die sich auch für neue Modelle der Billigautomarke empfiehlt.
\end{itemize}
}

\exercise{4|4}{Marketing-Mix (4 P's)}

Frau Schuberth verbringt heute einen Tag in der Marketing-Abteilung: 
\\~\\
\textit{"So, Frau Schuberth, um in Zukunft eine erfolgreiche Platzierung im Haushaltsbereich zu erreichen, müssen wir uns nun Gedanken um konkrete Maßnahmen rund um die Vermarktung unserer Kaffeemaschinen machen. Gegenüber unseren Kunden verfolgten wir bisher ein B2B Marketing. Ein Wechsel zur Zielgruppe Haushalt bedeutet auch eine Ausrichtung unseres Marketings zu B2C. Die Frage, die wir uns nun stellen müssen, lautet: Was ändert sich für unser Unternehmen in den 4P’s?"} 
\\~\\
Die OHM GmbH legt Wert auf eine hohe Produktqualität. Ihre Vollautomaten sind langlebige Produkte, die aber auch individuell für den Kunden angepasst werden können. Um den Qualitätsanspruch zu erhalten, sind die zusätzlichen Serviceleistungen immens wichtig. Reparaturen der Geräte sind zur gesamten Lebensdauer der Maschinen möglich. Die OHM GmbH verfolgt eine Hochpreisstrategie. 
\\~\\
\textit{"Also Frau Schuberth, unsere Kundinnen und Kunden bekommen von uns immer einen Kaufvertrag plus einen Servicevertrag für die gesamte Lebensdauer ihrer Maschine ausgestellt. Wir bieten keinerlei Rabattaktionen oder Preisdifferenzierungen, wie es im Consumer-Bereich üblich ist. Unsere Produkte sind komplexe und erklärungsbedürftige Geräte. Deswegen ist uns der direkte Kontakt zum Kunden wichtig. Unsere Sales Manager gehen vor allem auf Messen der Hotellerie und der Gastronomie, wo sie unsere Kunden antreffen werden. Wir pflegen auch eine starke Beziehung zu verschiedenen Röstereien, die wiederum unsere Maschinen an ihre Kunden weiterempfehlen. Der persönliche Verkauf und Beratung steht bei uns immer im Vordergrund. Deswegen betreiben wir auch unser Schulungscenter, wo unsere Kunden eine umfassende Beratung und ihre persönliche Schulung zu unseren Produkten genießen können. Darüber hinaus bieten wir dort weiterführende Kurse zur Kaffeezubereitung, um natürlich nebenbei unsere neuesten Innovationen anzubieten. Neben all den Offline-Aktivitäten müssen wir aber auch unser Online-Marketing beachten. Wir pflegen eine Firmen-Website, die unseren Qualitätsanspruch repräsentiert. In ansprechendem Design und intuitiver Navigation sind alle Informationen sofort sichtbar. Derzeit wird unsere Website für die Nutzung über mobile Endgeräte optimiert. Wir sind bisher nicht im Social Web tätig, wir kaufen auch keine Adwords, um uns bekannter zu machen. Wir betreiben auch keinen Shop, da der Endverbraucher bisher nicht unser Ziel war. Wir haben jedoch einen Login-Bereich, den unsere Kunden, unsere Sales Manager und unsere Service-Techniker nutzen können. Der Zugang erlaubt den Zugriff auf alle möglichen Produktunterlagen wie Werbebroschüre, Bedienungsanleitung oder Datenblatt. Ein Konfigurator hilft bei der Zusammenstellung der gewünschten Kaffeemaschine. Der Vertrieb kann die aktuellen Preislisten abrufen und die Techniker haben Zugriff auf sämtliche technische Daten."} 
\\~\\
B2C Marketing zeichnet sich aus durch:
\begin{itemize}
    \item Eine große Anzahl von anonymen Abnehmern
    \item Meist erfolgt der Kauf bei einem Vertriebspartner (indirekter Vertrieb über Händlernetzwerk)
    \item Schnelle und emotionale Kaufentscheidung durch den Kunden
    \item Der Kunde entscheidet meist alleine über den Kauf des Produktes
\end{itemize}
\noindent
\textbf{Aufgabe:} 
\\~\\
Was ändert sich für die OHM GmbH bei einer Ausrichtung auf Privatkunden? \\~\\
Stellen Sie die bisherigen 4P’s der OHM GmbH dar (B2B-Marketing) und stellen Sie diese einem B2C-Marketing-Mix gegenüber.

\solution{

\textbf{B2C Marketing zeichnet sich aus durch:}
\begin{itemize}
    \item Eine große Anzahl von anonymen Abnehmern
    \item Meist erfolgt der Kauf bei einem Vertriebspartner (indirekter Vertrieb über Händlernetzwerk)
    \item Schnelle und emotionale Kaufentscheidung durch den Kunden
    \item Der Kunde entscheidet meist alleine über den Kauf des Produktes
\end{itemize}

\textbf{Vergleich der 4P's zwischen B2B- und B2C-Marketing-Mix:}

\begin{tabularx}{\textwidth}{|X|X|X|}
\hline
\textbf{Bereich} & \textbf{B2B-Marketing} & \textbf{B2C-Marketing} \\
\hline
\textbf{Produktpolitik} &
Hohe Produktqualität, langlebige Produkte, Individualisierung/Anpassung, serviceorientierte Ausrichtung, zusätzliche Serviceleistungen, Reparaturen zur gesamten Produktlebensdauer &
Produktkauf viel emotionaler, kürzerer Produktlebenszyklus, Produktindividualisierung und After-Sales-Service spielen nur eine untergeordnete Rolle \\
\hline
\textbf{Preispolitik} &
Hochpreisstrategie, Kaufvertrag plus Servicevertrag &
Aktionsangebote, Rabatte, Preisdifferenzierungen \\
\hline
\textbf{Vertriebspolitik} &
Direktvertrieb &
Indirekter Vertrieb über Zwischenhändler \\
\hline
\textbf{Kommunikationspolitik} &
Below-the-line Promotion, komplexe, erklärungsbedürftige Geräte, persönlicher Verkauf steht im Vordergrund, direkter Kontakt zum Kunden, Besuch von Messen, wo Kunde und Anbieter zusammentreffen, Schulungscenter, Referenzkunden, Key Account Manager, Empfehlungen durch Röstereien, Firmen-Website: seriöse Darstellung für die Außenwelt plus Kunden-Login &
Above-the-line Promotion, Massenkommunikation über klassische Medien, Fokus auf Internet, TV, Print, emotional und einprägsam, höhere Markenbekanntheit, indirekter Verkauf über den Handel \\
\hline
\end{tabularx}
}

