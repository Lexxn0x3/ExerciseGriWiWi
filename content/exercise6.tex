\exercise{6|1}{Auszahlungen und Ausgaben}

\textbf{Aufgabe:}  
Geben Sie Beispiele, bei denen die Auszahlungen  
\begin{enumerate}[label=\alph*)]
    \item den Ausgaben entsprechen.
    \item den Ausgaben nicht entsprechen.
\end{enumerate}

\solution{
\textbf{a)}  
\begin{itemize}
    \item In diesem Fall nimmt der Zahlungsmittelbestand in gleichem Maße ab wie das Geldvermögen.
    \item Dies ist nur dann möglich, wenn – abgesehen vom Zahlungsmittelbestand – die übrigen Komponenten des Geldvermögens (Forderungen, Verbindlichkeiten) unverändert bleiben.
    \item \textbf{Beispiel:} Barkauf von Maschinen.
\end{itemize}

\textbf{b)}  
\begin{itemize}
    \item In diesem Fall steht der mit der Auszahlung verbundenen Verringerung des Zahlungsmittelbestandes eine betragsgleiche Erhöhung der Forderungen oder Verringerung der Verbindlichkeiten gegenüber, sodass sich das Geldvermögen nicht ändert.
    \item \textbf{Beispiel:} Rückzahlung eines in der Vorperiode aufgenommenen Kredites.
\end{itemize}
}
\exercise{6|2}{Gemeinkosten}

Welche Probleme ergeben sich im Zusammenhang mit den Gemeinkosten?

\solution{

\begin{itemize}
    \item Gemeinkosten können nicht direkt einer Bezugsgröße (meist Kostenträger) zugeordnet werden (\textit{echte Gemeinkosten}) oder ihre Zurechnung wäre zu aufwendig (\textit{unechte Gemeinkosten}).
    \item Die Umlage von Gemeinkosten auf die einzelnen Kostenträger ist vor allem problematisch, wenn der Verbrauch nicht in physischen Einheiten wie z. B. Stück, Liter, Kubikmeter, messbar ist.
    \item Es gibt demzufolge verschiedene Möglichkeiten der Kostenverrechnung, die jedoch alle nur bedingt wirklichkeitsgetreue Ergebnisse bereitstellen.
\end{itemize}

\begin{table}[h!]
    \centering
    \renewcommand{\arraystretch}{1.5}
    \begin{tabularx}{\textwidth}{|l|X|}
        \hline
        \textbf{Verursachungsprinzip} & Verteilung der Gemeinkosten auf die Bezugsgrößen gemäß ihrer Verursachung. Hier wird zwar eine wirklichkeitsgetreue Abbildung des Kostenanfalls erreicht, dennoch ist die Durchführbarkeit insbesondere in Bezug auf die Gemeinkosten sehr eingeschränkt. \\
        \hline
        \textbf{Tragfähigkeitsprinzip} & Dabei werden die Kosten im Verhältnis der Preise oder Stückdeckungsbeiträge auf die Kostenträger umgelegt, wobei der Kostenanteil proportional zum Preis/Stückdeckungsbeitrag steigt. Nachteil dieser Methode ist die nicht realitätsnahe Aufteilung der Kosten. \\
        \hline
        \textbf{Durchschnittsprinzip} & Die Gemeinkosten werden durchschnittlich den Kostenträgern zugerechnet. Hier wird weder eine realitätsgetreue noch eine wirtschaftlichkeitsbezogene Darstellung der Leistungssituation erreicht. \\
        \hline
    \end{tabularx}
\end{table}
}

\solution{
\begin{enumerate}[label=\alph*)]
    \item \textbf{Gesamtkosten für die Produktvariante STANDARD:}
    \[
        \text{Einzelkosten pro Stück} = 120,00\,€ + 5,00\,€ + 40,00\,€ = 165,00\,€
    \]
    \[
        \text{Gesamtkosten für 1.000 Stück} = 1.000 \cdot 165,00\,€ = 165.000,00\,€
    \]
    \[
        \text{Fixkostenanteil pro Jahr} = \frac{5.000,00\,€ + 10.200,00\,€}{2} \cdot 12 = 91.200,00\,€
    \]
    \[
        \text{Gesamtkosten} = 165.000,00\,€ + 91.200,00\,€ = 256.200,00\,€
    \]

    \item \textbf{Gesamtkosten für die Produktvariante PREMIUM:}
    \[
        \text{Einzelkosten pro Stück} = 136,00\,€ + 64,00\,€ + 30,00\,€ = 230,00\,€
    \]
    \[
        \text{Gesamtkosten für 800 Stück} = 800 \cdot 230,00\,€ = 184.000,00\,€
    \]
    \[
        \text{Fixkostenanteil pro Jahr} = \frac{5.000,00\,€ + 10.200,00\,€}{2} \cdot 12 = 91.200,00\,€
    \]
    \[
        \text{Gesamtkosten} = 184.000,00\,€ + 91.200,00\,€ = 275.200,00\,€
    \]

    \item \textbf{Unternehmensgewinn:}
    \[
        \text{Umsatz STANDARD} = 1.000 \cdot 300,00\,€ = 300.000,00\,€
    \]
    \[
        \text{Umsatz PREMIUM} = 800 \cdot 450,00\,€ = 360.000,00\,€
    \]
    \[
        \text{Gesamtumsatz} = 300.000,00\,€ + 360.000,00\,€ = 660.000,00\,€
    \]
    \[
        \text{Gesamtkosten} = 256.200,00\,€ + 275.200,00\,€ = 531.400,00\,€
    \]
    \[
        \text{Gewinn} = 660.000,00\,€ - 531.400,00\,€ = 128.600,00\,€
    \]
\end{enumerate}
}

\exercise{6|4}{Fixe und variable Kosten}

In einer Zeitschrift machte ein Manager eines Industriebetriebes folgende Aussage:  
„Tendenziell gibt es immer weniger variable Kosten und dafür immer mehr fixe Kosten.“  

\begin{enumerate}[label=\alph*)]
    \item Was meint der Manager damit?
    \item Wie können Sie sich diesen Sachverhalt erklären?
\end{enumerate}

\solution{
\begin{enumerate}[label=\alph*)]
    \item \textbf{Erklärung der Aussage:}  
    Der Manager beschreibt eine allgemeine Entwicklung in der Industrie, bei der die Bedeutung der fixen Kosten im Vergleich zu den variablen Kosten zunimmt. Dies bedeutet, dass die Unternehmen immer stärker in langfristige, feste Investitionen wie Maschinen, Anlagen und Technologien investieren, während die variablen Kosten, die direkt mit der Produktionsmenge zusammenhängen (z. B. Material- oder Lohnkosten), im Verhältnis abnehmen.

    \item \textbf{Erklärung des Sachverhalts:}  
    Dieser Sachverhalt lässt sich durch folgende Gründe erklären:
    \begin{itemize}
        \item \textbf{Automatisierung und Digitalisierung:}  
        Durch den Einsatz moderner Technologien und Automatisierung sinken die arbeitsintensiven variablen Kosten (z. B. Löhne), während die fixen Investitionen in Maschinen und Systeme steigen.
        \item \textbf{Skaleneffekte:}  
        Unternehmen versuchen, Skaleneffekte zu nutzen, indem sie größere Produktionsmengen mit geringeren variablen Kosten pro Einheit erreichen. Dies führt zu einer stärkeren Gewichtung der fixen Kosten.
        \item \textbf{Investitionen in Forschung und Entwicklung:}  
        Unternehmen investieren zunehmend in F\&E, um wettbewerbsfähig zu bleiben. Diese Ausgaben sind fix und unabhängig von der Produktionsmenge.
        \item \textbf{Verlagerung der Produktion:}  
        Produktionsprozesse werden zunehmend in Länder mit niedrigeren variablen Kosten verlagert, wodurch die verbleibenden Kostenstrukturen stärker von fixen Kosten geprägt sind.
    \end{itemize}
\end{enumerate}
}

\exercise{6|4}{Fixe und variable Kosten}

In einer Zeitschrift machte ein Manager eines Industriebetriebes folgende Aussage:  
„Tendenziell gibt es immer weniger variable Kosten und dafür immer mehr fixe Kosten.“  

\begin{enumerate}[label=\alph*)]
    \item Was meint der Manager damit?
    \item Wie können Sie sich diesen Sachverhalt erklären?
\end{enumerate}

\solution{
\begin{enumerate}[label=\alph*)]
    \item \textbf{Erklärung der Aussage:}  
    Der Manager beschreibt eine allgemeine Entwicklung in der Industrie, bei der die Bedeutung der fixen Kosten im Vergleich zu den variablen Kosten zunimmt. Dies bedeutet, dass die Unternehmen immer stärker in langfristige, feste Investitionen wie Maschinen, Anlagen und Technologien investieren, während die variablen Kosten, die direkt mit der Produktionsmenge zusammenhängen (z. B. Material- oder Lohnkosten), im Verhältnis abnehmen.

    \item \textbf{Erklärung des Sachverhalts:}  
    Dieser Sachverhalt lässt sich durch folgende Gründe erklären:
    \begin{itemize}
        \item \textbf{Automatisierung und Digitalisierung:}  
        Durch den Einsatz moderner Technologien und Automatisierung sinken die arbeitsintensiven variablen Kosten (z. B. Löhne), während die fixen Investitionen in Maschinen und Systeme steigen.
        \item \textbf{Skaleneffekte:}  
        Unternehmen versuchen, Skaleneffekte zu nutzen, indem sie größere Produktionsmengen mit geringeren variablen Kosten pro Einheit erreichen. Dies führt zu einer stärkeren Gewichtung der fixen Kosten.
        \item \textbf{Investitionen in Forschung und Entwicklung:}  
        Unternehmen investieren zunehmend in Forschung und Entwicklung (F\&E), um wettbewerbsfähig zu bleiben. Diese Ausgaben sind fix und unabhängig von der Produktionsmenge.
        \item \textbf{Verlagerung der Produktion:}  
        Produktionsprozesse werden zunehmend in Länder mit niedrigeren variablen Kosten verlagert, wodurch die verbleibenden Kostenstrukturen stärker von fixen Kosten geprägt sind.
    \end{itemize}
\end{enumerate}
}

