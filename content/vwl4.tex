\exercise{4\textbar 1}{Deutsche Vereinigung von 1990}
Die in diesem Kapitel dargestellte Theorie der internationalen Arbeitsteilung besagt, dass es für eine erfolgreiche Arbeitsteilung \textbf{nicht} auf das absolute Niveau der Produktivität eines Landes ankommt.

\begin{enumerate}[label=(\alph*)]
    \item Erklären Sie die Theorie der internationalen Arbeitsteilung.
    \item Wieso konnte sich die Industrie in Ostdeutschland nach der Wiedervereinigung nicht mehr behaupten?
\end{enumerate}

\solution{

\begin{itemize}
    \item Die in diesem Kapitel dargestellte \textbf{Theorie der internationalen Arbeitsteilung} besagt, dass es für eine erfolgreiche Arbeitsteilung nicht auf das absolute Niveau der Produktivität eines Landes ankommt.
    \item Da also die absoluten Kostenvorteile für die Arbeitsteilung ohne Bedeutung sind, konnten sich Länder wie Polen oder Ungarn im internationalen Wettbewerb behaupten, auch wenn ihre Produktivität bei der Herstellung aller Güter geringer ist als die ihrer Konkurrenten im Westen.
    \item Entscheidend für die Arbeitsteilung sind allein die \textbf{komparativen Kostenvorteile}.
    \item Das bedeutet, ein Land kann international erfolgreich sein, wenn es bei der Produktion eines Gutes A im Vergleich zum Ausland \textbf{relativ produktiver} ist als bei der Herstellung eines Gutes B.
    \item In Polen und Tschechien sind dies vor allem \textbf{Güter mit geringerer Technologie} (z. B. Möbel, Textilien).
    \item Nach diesem Theorem hätte nun \textbf{eigentlich auch die Industrie in Ostdeutschland} in der Lage sein müssen, sich auf dem Weltmarkt zu behaupten.
    \item Dabei ist jedoch eine wichtige \textbf{Nebenbedingung} für die Anwendung des Theorems der komparativen Kosten zu berücksichtigen.
    \item Es kommt dabei in der Praxis nur dann zum Warenaustausch, wenn die Güter aus den Ländern mit niedriger Produktivität \textbf{auch vom Preis her wettbewerbsfähig} sind.
    \item Dazu ist es erforderlich, dass dort das \textbf{Lohnniveau insgesamt unter dem Niveau der Länder mit höherer Produktivität} liegt, wobei die Unterschiede in den Lohnniveaus in etwa den Unterschieden in der absoluten Produktivität entsprechen müssen.
    \item Konkret lagen im Jahr 2001 die \textbf{Arbeitskosten pro Stunde in Polen bei 3,76 Euro} und in Tschechien bei \textbf{3,30 Euro}.
    \item In Ostdeutschland kostet die Arbeitsstunde \textbf{18,86 Euro} und ist damit nicht mehr sehr weit vom Niveau in Westdeutschland entfernt, wo für eine Arbeitsstunde ein Preis von \textbf{26,16 Euro} zu zahlen ist.
    \item Da also in Polen und Tschechien die Löhne nur etwa \textbf{15 \% des westdeutschen Niveaus} ausmachen, sind dort also Unternehmen noch wettbewerbsfähig geblieben, deren Produktivität bei beispielsweise etwa \textbf{20 \% der westdeutschen Produktivität} liegt.
    \item In Ostdeutschland ging man 1990 davon aus, dass die Unternehmen etwa \textbf{ein Drittel ihrer westdeutschen Konkurrenten} aufweisen.
    \item Bei dem rasch erreichten hohen Lohnniveau in den neuen Bundesländern war daher \textbf{ein Großteil der Unternehmen nicht mehr wettbewerbsfähig}.
\end{itemize}
}
\exercise{4\textbar 2}{Internationale Arbeitsteilung}
In A-Land werden Handys und Hemden produziert. Zur Herstellung eines Handys wird eine Arbeitszeit von 30 Minuten benötigt, für ein Hemd sind es 20 Minuten.

In B-Land werden diese beiden Produkte ebenfalls hergestellt. Hier liegt die Arbeitszeit für ein Hemd und ein Handy bei jeweils 15 Minuten.

\begin{enumerate}[label=(\alph*)]
    \item Erläutern Sie anhand des Beispiels das Konzept der absoluten und der komparativen Kostenvorteile!
    \item Welche Effekte ergeben sich für die beiden Länder, wenn sie miteinander Außenhandel betreiben?
    \item Nehmen Sie zur Vereinfachung an, die insgesamt verfügbare Arbeitszeit liege in beiden Ländern bei jeweils 100 Stunden. Zeichnen Sie die individuellen Transformationskurven und die Transformationskurve bei effizienter Arbeitsteilung!
\end{enumerate}

\solution{
\textbf{a):}

Die absoluten Kostenvorteile geben an, welches Land bei der Herstellung der beiden Güter jeweils produktiver ist. Dies kann man ermitteln, indem man sich fragt, wie viel Zeit in A-Land und in B-Land zur Herstellung benötigt wird (siehe Tabelle unten).

\begin{table}[H]
    \centering
    \begin{tabular}{|l|c|c|}
        \hline
        \textbf{Produkt} & \textbf{A-Land (Minuten)} & \textbf{B-Land (Minuten)} \\
        \hline
        Handys & 30 & 15 \\
        \hline
        Hemden & 20 & 15 \\
        \hline
    \end{tabular}
    \label{tab:zeitbedarf}
\end{table}

Aus der Tabelle wird ersichtlich, dass B-Land sowohl bei der Produktion von Handys als auch bei der Produktion von Hemden absolute Kostenvorteile aufweist, da es in beiden Fällen weniger Zeit benötigt.

\textbf{Komparative Kostenvorteile:}

Der Verbrauchskoeffizient (Zeitbedarf zur Herstellung einer Einheit eines Gutes) ist bei beiden Gütern in B-Land geringer als in A-Land. Bei den komparativen Kosten geht es darum, auf wie viele Einheiten an Handys (Hemden) man in den beiden Ländern verzichten muss, wenn man eine zusätzliche Einheit des anderen Gutes herstellen möchte.

Für A-Land ergibt sich:
\[
\text{Kosten eines zusätzlichen Handys} = \frac{300 \, \text{Hemden}}{200 \, \text{Handys}} = \frac{3}{2} \, \text{Hemden}
\]
\[
\text{Kosten eines zusätzlichen Hemdes} = \frac{200 \, \text{Handys}}{300 \, \text{Hemden}} = \frac{2}{3} \, \text{Handy}
\]

Für B-Land ergibt sich analog:
\[
\text{Kosten eines zusätzlichen Handys} = 1 \, \text{Hemd}
\]
\[
\text{Kosten eines zusätzlichen Hemdes} = 1 \, \text{Handy}
\]

\begin{table}[H]
    \centering
    \begin{tabular}{|l|c|c|}
        \hline
        \textbf{Kosten} & \textbf{A-Land} & \textbf{B-Land} \\
        \hline
        Für ein zusätzliches Handy & 3/2 Hemden & 1 Hemd \\
        \hline
        Für ein zusätzliches Hemd & 2/3 Handy & 1 Handy \\
        \hline
    \end{tabular}
    \label{tab:komparative_kosten}
\end{table}

\textbf{Fazit:}
\begin{itemize}
    \item \textbf{A-Land} hat einen komparativen Kostenvorteil bei der Produktion von \textbf{Hemden}.
    \item \textbf{B-Land} hat einen komparativen Kostenvorteil bei der Produktion von \textbf{Handys}.
\end{itemize}





\textbf {b)}
Außenhandel ermöglicht es jedem Land, sich auf das Gut zu spezialisieren, das es relativ am effizientesten produzieren kann:
\begin{itemize}
    \item \textbf{A-Land} wird sich also auf die Produktion von Hemden festlegen.
    \item \textbf{B-Land} wird primär Handys herstellen.
\end{itemize}

Durch den Handel verbessert sich die Güterausstattung beider Länder im Vergleich zum Zustand ohne Außenhandel. Außenhandel erlaubt eine Win-Win-Situation, indem beide Länder von den jeweiligen komparativen Vorteilen profitieren.

\textbf{c)}
\textit{Individuelle Transformationskurven}
\begin{itemize}
    \item Für \textbf{A-Land}: Die maximale Arbeitszeit von 100 Stunden (6.000 Minuten) kann entweder für 200 Handys (30 Minuten pro Handy) oder 300 Hemden (20 Minuten pro Hemd) genutzt werden. Die Transformationskurve ist eine gerade Linie, die diese beiden Extrempunkte verbindet.
    \item Für \textbf{B-Land}: Mit denselben 100 Stunden (6.000 Minuten) können entweder 400 Handys (15 Minuten pro Handy) oder 400 Hemden (15 Minuten pro Hemd) produziert werden. Auch hier verbindet eine gerade Linie diese Extrempunkte.
\end{itemize}

\textit{Gemeinsame Transformationskurve}
Die gemeinsame Transformationskurve ergibt sich durch eine optimale Kombination der komparativen Vorteile beider Länder:
\begin{itemize}
    \item Zunächst wird A-Land vollständig auf die Hemdenproduktion spezialisiert, da es hierbei einen komparativen Vorteil hat.
    \item B-Land wird vollständig Handys produzieren, da es bei Handys den komparativen Vorteil aufweist.
    \item Die gemeinsame Transformationskurve setzt sich aus den maximalen Werten beider Länder zusammen. Sobald A-Land die maximale Menge an Hemden produziert hat (300 Hemden), beginnt B-Land mit der Produktion der restlichen Hemden und Handys.
\end{itemize}

Durch diese Arbeitsteilung und den darauf aufbauenden Außenhandel können beide Länder ihre Gesamtproduktion optimieren.

}




\exercise{4\textbar 3}{Schreinerei Hartholz}

Meister Hartholz hat zwei Gesellen, den Willi und den Franz.  
Die Schreinerei hat sich darauf spezialisiert, Fenster und Türen herzustellen.  
Meister Hartholz steht nun vor der Frage, wie er seine beiden Gesellen in der Herstellung dieser beiden Produkte einsetzen soll.

Er hat ermittelt, dass Willi in einer Woche maximal 60 Türen oder aber 100 Fenster herstellen kann. Franz ist bei Weitem nicht so geschickt und bringt es nur auf 50 Türen oder 50 Fenster (siehe Tabelle \ref{tab:output}).  

Wie es sich für ein Modell gehört, nehmen wir jetzt einfach an, dass es nicht möglich ist, eine Tür oder ein Fenster von Willi und Franz in Gemeinschaftsarbeit herzustellen.

Nun erhält Meister Hartholz einen Auftrag über 55 Türen und 80 Fenster, der in einer Woche erledigt werden muss.  
Er hat dazu zunächst einmal drei Optionen durchgerechnet:

\begin{enumerate}[label=(\alph*)]
    \item Willi produziert nur Türen, Franz nur Fenster.
    \item Willi stellt die Hälfte der Woche Türen her, die andere Fenster. Franz macht das genauso.
    \item Willi konzentriert sich auf Fenster, Franz auf Türen.
\end{enumerate}

In seinem Notizbuch hat der Meister dann folgende Tabelle (Tabelle \ref{tab:auftragsbuch}) erstellt:  
Er kommt so zu dem unschönen Ergebnis, dass er mit keiner der Optionen in der Lage ist, den Auftrag fristgerecht zu erfüllen. Sein Sohn, der in der nahegelegenen Universitätsstadt Volkswirtschaftslehre studiert, schlägt ihm jedoch eine Lösung nach dem Prinzip der komparativen Kosten vor, mit der er alles fristgerecht erledigen kann.  

\textbf{Wie muss Herr Hartholz dann vorgehen?}

\begin{table}[h!]
    \centering
    \begin{tabular}{|c|c|c|}
        \hline
        & \textbf{Türen} & \textbf{Fenster} \\
        \hline
        Willi & 60 & 100 \\
        \hline
        Franz & 50 & 50 \\
        \hline
    \end{tabular}
    \caption{Wöchentlicher Output von Franz und Willi}
    \label{tab:output}
\end{table}

\begin{table}[h!]
    \centering
    \begin{tabular}{|c|c|c|}
        \hline
        \textbf{Option} & \textbf{Türen} & \textbf{Fenster} \\
        \hline
        Option A & 60 & 50 \\
        \hline
        Option B & 55 & 75 \\
        \hline
        Option C & 50 & 100 \\
        \hline
    \end{tabular}
    \caption{Auftragsbuch von Meister Hartholz}
    \label{tab:auftragsbuch}
\end{table}
\solution{

\begin{itemize}
    \item \textbf{Komparative Kostenvorteile} liegen vor, wenn eine Person oder ein Land Gut 1 effizienter produzieren kann als Gut 2 im Vergleich zu einer anderen.
    \item Das \textbf{Prinzip der komparativen Kostenvorteile} besagt also:
    Jeder sollte sich auf das spezialisieren, was er relativ am effizientesten produzieren kann.
    \item Um die komparativen Kosten von Franz und Willi zu ermitteln, muss man die Tabelle so umformulieren, dass sie zeigt, auf wie viele Einheiten von Fenstern bzw. Türen Franz und Willi jeweils verzichten müssen, wenn sie eine zusätzliche Tür bzw. ein zusätzliches Fenster produzieren:
\end{itemize}

\begin{table}[h!]
    \centering
    \begin{tabular}{|c|c|c|}
        \hline
        & \textbf{Kosten eines zusätzlich produzierten Fensters} & \textbf{Kosten einer zusätzlich produzierten Tür} \\
        \hline
        Willi & 0,6 Türen & 5/3 Fenster \\
        \hline
        Franz & 1 Tür & 1 Fenster \\
        \hline
    \end{tabular}
    \caption{Komparative Kosten von Franz und Willi}
    \label{tab:komparative_kosten}
\end{table}

\textbf{Schlussfolgerungen:}
\begin{itemize}
    \item Willi hat \textbf{komparative Vorteile} bei der Produktion von Fenstern.
    \item Franz hat \textbf{komparative Kostenvorteile} bei der Produktion von Türen.
    \item Nebenbei bemerkt sieht man in der Tabelle, dass Willi \textbf{absolute Kostenvorteile} gegenüber Franz bei beiden Gütern aufweist, der in einer Woche sowohl mehr Türen als auch mehr Fenster produzieren kann.
\end{itemize}

\textbf{Lösung für Meister Hartholz:}
\begin{itemize}
    \item Die Lösung muss darin bestehen, dass sich zunächst Willi auf die Fensterproduktion spezialisiert.
    \item Er wird dann in der einen Woche alle 80 bestellten Fenster herstellen und hat danach noch ein Fünftel seiner Wochen-Arbeitszeit verfügbar.
    \item In dieser Zeit kann er noch 12 Türen herstellen.
    \item Franz spezialisiert sich ganz auf die Türen und kann in der Woche 50 Türen herstellen.
\end{itemize}

\textbf{Ergebnis:}  
In der Summe erbringt also die Arbeitsteilung nach dem Prinzip der komparativen Kosten 80 Fenster und 62 Türen, womit der Auftrag (55 Türen und 80 Fenster) problemlos abgewickelt werden kann.
}

