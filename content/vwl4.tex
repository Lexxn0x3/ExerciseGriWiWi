\exercise{4\textbar 1}{Deutsche Vereinigung von 1990}
Die in diesem Kapitel dargestellte Theorie der internationalen Arbeitsteilung besagt, dass es für eine erfolgreiche Arbeitsteilung \textbf{nicht} auf das absolute Niveau der Produktivität eines Landes ankommt.

\begin{enumerate}[label=(\alph*)]
    \item Erklären Sie die Theorie der internationalen Arbeitsteilung.
    \item Wieso konnte sich die Industrie in Ostdeutschland nach der Wiedervereinigung nicht mehr behaupten?
\end{enumerate}

\solution{

\begin{itemize}
    \item Die in diesem Kapitel dargestellte \textbf{Theorie der internationalen Arbeitsteilung} besagt, dass es für eine erfolgreiche Arbeitsteilung nicht auf das absolute Niveau der Produktivität eines Landes ankommt.
    \item Da also die absoluten Kostenvorteile für die Arbeitsteilung ohne Bedeutung sind, konnten sich Länder wie Polen oder Ungarn im internationalen Wettbewerb behaupten, auch wenn ihre Produktivität bei der Herstellung aller Güter geringer ist als die ihrer Konkurrenten im Westen.
    \item Entscheidend für die Arbeitsteilung sind allein die \textbf{komparativen Kostenvorteile}.
    \item Das bedeutet, ein Land kann international erfolgreich sein, wenn es bei der Produktion eines Gutes A im Vergleich zum Ausland \textbf{relativ produktiver} ist als bei der Herstellung eines Gutes B.
    \item In Polen und Tschechien sind dies vor allem \textbf{Güter mit geringerer Technologie} (z. B. Möbel, Textilien).
    \item Nach diesem Theorem hätte nun \textbf{eigentlich auch die Industrie in Ostdeutschland} in der Lage sein müssen, sich auf dem Weltmarkt zu behaupten.
    \item Dabei ist jedoch eine wichtige \textbf{Nebenbedingung} für die Anwendung des Theorems der komparativen Kosten zu berücksichtigen.
    \item Es kommt dabei in der Praxis nur dann zum Warenaustausch, wenn die Güter aus den Ländern mit niedriger Produktivität \textbf{auch vom Preis her wettbewerbsfähig} sind.
    \item Dazu ist es erforderlich, dass dort das \textbf{Lohnniveau insgesamt unter dem Niveau der Länder mit höherer Produktivität} liegt, wobei die Unterschiede in den Lohnniveaus in etwa den Unterschieden in der absoluten Produktivität entsprechen müssen.
    \item Konkret lagen im Jahr 2001 die \textbf{Arbeitskosten pro Stunde in Polen bei 3,76 Euro} und in Tschechien bei \textbf{3,30 Euro}.
    \item In Ostdeutschland kostet die Arbeitsstunde \textbf{18,86 Euro} und ist damit nicht mehr sehr weit vom Niveau in Westdeutschland entfernt, wo für eine Arbeitsstunde ein Preis von \textbf{26,16 Euro} zu zahlen ist.
    \item Da also in Polen und Tschechien die Löhne nur etwa \textbf{15 \% des westdeutschen Niveaus} ausmachen, sind dort also Unternehmen noch wettbewerbsfähig geblieben, deren Produktivität bei beispielsweise etwa \textbf{20 \% der westdeutschen Produktivität} liegt.
    \item In Ostdeutschland ging man 1990 davon aus, dass die Unternehmen etwa \textbf{ein Drittel ihrer westdeutschen Konkurrenten} aufweisen.
    \item Bei dem rasch erreichten hohen Lohnniveau in den neuen Bundesländern war daher \textbf{ein Großteil der Unternehmen nicht mehr wettbewerbsfähig}.
\end{itemize}
}

