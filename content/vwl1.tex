\section{VWL}
\exercise{1\textbar 1}{Open-Book-Prinzip}
In manchen Klausuren wird nach dem Prinzip des „Open Book“ vorgegangen, d. h. die Studierenden können ihre Lehrbücher mit in die Klausur nehmen. 

Student Hubert findet, das ist eine tolle Sache. Er kann es überhaupt nicht verstehen, dass seine Kommilitonin Sarah es lieber hätte, wenn die Bücher zu Hause gelassen werden müssten.

\begin{quote}
    Wer hat Recht?
\end{quote}

\solution{

Beide Ansichten spiegeln unterschiedliche Herangehensweisen an die Prüfungsstrategie wider, aber das zugrunde liegende Problem könnte das sogenannte **Konkurrenz-Paradoxon** sein:
\begin{itemize}
    \item \textbf{Pro Open-Book (Hubert):}
    \begin{itemize}
        \item Hubert freut sich über den Zugang zu Lehrbüchern, da sie ihm die Möglichkeit geben, schwierige Details nachzuschlagen und seine Konzentration auf das Verständnis und die Anwendung zu legen.
        \item Das Open-Book-Prinzip kann Prüfungsstress reduzieren, indem Studierende nicht alles auswendig lernen müssen.
    \end{itemize}
    
    \item \textbf{Kontra Open-Book (Sarah):}
    \begin{itemize}
        \item Sarah könnte befürchten, dass durch das Open-Book-Prinzip alle Studierenden die gleiche "Hilfestellung" haben, wodurch das individuelle Verständnis weniger ins Gewicht fällt.
        \item Das Paradoxon besteht darin, dass alle Studierenden theoretisch Zugang zu den gleichen Ressourcen haben. Der Wettbewerbsvorteil verschwindet, und es kommt stärker darauf an, wer besser mit den Ressourcen umgehen kann.
    \end{itemize}
\end{itemize}

\textbf{Fazit:} 
Das Open-Book-Prinzip kann helfen, den Fokus auf Verständnis und Problemlösungsfähigkeiten zu legen. Es stellt jedoch andere Anforderungen an die Studierenden, wie z. B. die Fähigkeit, effizient relevante Informationen zu finden. Es bleibt ein Balanceakt zwischen Chancengleichheit und der Fähigkeit, aus gleichen Ressourcen Vorteile zu ziehen.
}

\exercise{1\textbar 2}{Mikro- und Makroökonomie}
Mikro- und Makroökonomie sind die beiden zentralen Teilgebiete der Volkswirtschaftslehre.

Ordnen Sie die folgenden Fragestellungen in das entsprechende Teilgebiet ein:
\begin{enumerate}
    \item Ist das Zinsniveau in Euroland derzeit zu hoch?
    \item Sollte die Deutsche Post wieder vom Staat betrieben werden?
    \item Soll sich die Bundesregierung bemühen, weiterhin einen ausgeglichenen Haushalt vorzulegen?
    \item Ist es richtig, dass die Europäische Union die Landwirtschaft subventioniert?
    \item Sollte die Leiharbeit in Deutschland ausgeweitet werden?
    \item Sollte für Studentenwohnungen ein Höchstpreis eingeführt werden?
    \item Welche Auswirkungen hat ein Kurseinbruch an den Aktienmärkten?
    \item Soll die Ökosteuer noch weiter erhöht werden?
    \item Kommt es im nächsten Jahr zu einem kräftigen Wirtschaftswachstum in Deutschland?
    \item Soll der Zahnersatz von den Gesetzlichen Krankenkassen vollständig bezahlt werden?
\end{enumerate}

\solution{

\begin{itemize}
    \item \textbf{Mikroökonomie:} Untersuchung einzelner Märkte, Entscheidungen einzelner Akteure und deren Einfluss auf Angebot und Nachfrage.
    \begin{itemize}
        \item 2. Sollte die Deutsche Post wieder vom Staat betrieben werden?
        \item 5. Sollte die Leiharbeit in Deutschland ausgeweitet werden?
        \item 6. Sollte für Studentenwohnungen ein Höchstpreis eingeführt werden?
        \item 10. Soll der Zahnersatz von den Gesetzlichen Krankenkassen vollständig bezahlt werden?
    \end{itemize}

    \item \textbf{Makroökonomie:} Untersuchung gesamtwirtschaftlicher Zusammenhänge wie Inflation, Arbeitslosigkeit und Wirtschaftswachstum.
    \begin{itemize}
        \item 1. Ist das Zinsniveau in Euroland derzeit zu hoch?
        \item 3. Soll sich die Bundesregierung bemühen, weiterhin einen ausgeglichenen Haushalt vorzulegen?
        \item 4. Ist es richtig, dass die Europäische Union die Landwirtschaft subventioniert?
        \item 7. Welche Auswirkungen hat ein Kurseinbruch an den Aktienmärkten?
        \item 8. Soll die Ökosteuer noch weiter erhöht werden?
        \item 9. Kommt es im nächsten Jahr zu einem kräftigen Wirtschaftswachstum in Deutschland?
    \end{itemize}
\end{itemize}
}

