\exercise{5\textbar 1}{Die vier Möglichkeiten der Finanzierung}
Wie kann sich ein Unternehmen grundsätzlich finanzieren? Geben Sie für jede Finanzierungsmöglichkeit zwei Beispiele an!

\solution{
Ein Unternehmen hat grundsätzlich vier Finanzierungsmöglichkeiten:
\begin{enumerate}
    \item \textbf{Erhöhung von Einzahlungen} (z.\,B. mehr Verkäufe, Aufnahme zusätzlicher Kredite).
    \item \textbf{Frühere Eingänge von Einzahlungen} (z.\,B. Vermeidung von Zahlungsfristen bei den Kunden, Erhöhung der Kundenanzahlungen).
    \item \textbf{Vermeidung von Auszahlungen} (z.\,B. Einsparungen durch Kostensenkungsmaßnahmen, Durchsetzung höherer Rabatte bei den Lieferanten).
    \item \textbf{Hinauszögerung von Auszahlungen} (z.\,B. spätere Bezahlung der Lieferanten, Tilgungsaussetzung bei Krediten).
\end{enumerate}
}

\exercise{5\textbar 2}{Rentabilitätskennzahlen}
Für ein Unternehmen gelten in einem Jahr folgende Daten:

\begin{table}[H]
\centering
\begin{tabular}{|l|r|}
\hline
\textbf{Durchschnittliches Eigenkapital} & 2.500.000 € \\
\textbf{Durchschnittliches Fremdkapital} & 3.500.000 € \\
\textbf{Gewinn} & 300.000 € \\
\textbf{Umsatz} & 12.500.000 € \\
\textbf{Fremdkapitalzinssatz} & 8 \% \\
\hline
\end{tabular}
\end{table}

\begin{enumerate}[label=(\alph*)]
    \item Wie hoch sind Eigenkapitalrentabilität?
    \item Wie hoch sind Gesamtkapitalrentabilität?
    \item Wie hoch sind Umsatzrentabilität?
\end{enumerate}

\solution{
\textbf{Lösung: Rentabilitätskennzahlen}

Für ein Unternehmen gelten in einem Jahr folgende Daten:
\begin{table}[H]
\centering
\begin{tabular}{|l|r|}
\hline
\textbf{Durchschnittliches Eigenkapital} & 2.500.000 € \\
\textbf{Durchschnittliches Fremdkapital} & 3.500.000 € \\
\textbf{Gewinn} & 300.000 € \\
\textbf{Umsatz} & 12.500.000 € \\
\textbf{Fremdkapitalzinssatz} & 8 \% \\
\hline
\end{tabular}
\end{table}

\begin{enumerate}[label=(\alph*)]
    \item \textbf{Eigenkapitalrentabilität} \\
    Die Eigenkapitalrentabilität wird berechnet als:
    \[
    \textbf{Eigenkapitalrentabilität} = \frac{Gewinn}{Eigenkapital}
    \]
    \[
    \textbf{Eigenkapitalrentabilität} = \frac{300.000}{2.500.000} = 0{,}12 = 12\,\%.
    \]

    \item \textbf{Gesamtkapitalrentabilität} \\
    Die Gesamtkapitalrentabilität wird berechnet als:
    \[
    Gesamtkapitalrentabilität \textbf{(Return-on-ivestment)} = \frac{Gewinn + Fremdkapitalzinsen}{Gesamtkapital}
    \]
    Die Fremdkapitalzinsen betragen:
    \[
    Fremdkapitalzinsen = 3.500.000 \cdot 0{,}08 = 280.000\,€.
    \]
    Das Gesamtkapital beträgt:
    \[
    Gesamtkapital = 2.500.000 + 3.500.000 = 6.000.000\,€.
    \]
    Daraus ergibt sich:
    \[
    Gesamtkapitalrentabilität = \frac{300.000 + 280.000}{6.000.000} = 0{,}0967 = 9{,}67\,\%.
    \]

    \item \textbf{Umsatzrentabilität} \\
    Die Umsatzrentabilität wird berechnet als:
    \[
    Umsatzrentabilität = \frac{Gewinn}{Umsatz}
    \]
    \[
    Umsatzrentabilität = \frac{300.000}{12.500.000} = 0{,}024 = 2{,}4\,\%.
    \]
\end{enumerate}
}

\exercise{5\textbar 3}{Nettobarwertmethode / Berechnung des Nettobarwerts}
Ein Kaffeemaschinenhersteller erwägt für die Entwicklung einer neuen Kaffeemaschinenversion eine größere Investition für das Jahr 2022. Die erwarteten Ein- und Auszahlungen in den nächsten fünf Jahren werden wie folgt geschätzt:

\begin{table}[H]
\centering
\begin{tabular}{|c|r|r|r|r|r|r|}
\hline
\textbf{Jahr} & \textbf{2023} & \textbf{2024} & \textbf{2025} & \textbf{2026} & \textbf{2027} & \textbf{2028} \\
\hline
\textbf{Betrag (€)} & -200.000 & +37.000 & +50.000 & +56.000 & +60.000 & +64.000 \\
\hline
\textbf{Barwertfaktor 5 \%} & 1,00000 & 0,95240 & 0,90700 & 0,86380 & 0,82270 & 0,78350 \\
\hline
\textbf{Barwertfaktor 6 \%} & 1,00000 & 0,94340 & 0,89000 & 0,83960 & 0,79210 & 0,74730 \\
\hline
\textbf{Barwertfaktor 7 \%} & 1,00000 & 0,93460 & 0,87340 & 0,81630 & 0,76290 & 0,71300 \\
\hline
\textbf{Barwertfaktor 8 \%} & 1,00000 & 0,92590 & 0,85730 & 0,79380 & 0,73500 & 0,68060 \\
\hline
\textbf{Barwertfaktor 9 \%} & 1,00000 & 0,91740 & 0,84170 & 0,77220 & 0,70840 & 0,64990 \\
\hline
\textbf{Barwertfaktor 10 \%} & 1,00000 & 0,90910 & 0,82640 & 0,75130 & 0,68300 & 0,62090 \\
\hline
\end{tabular}
\end{table}

\begin{enumerate}[label=(\alph*)]
    \item Wie hoch ist der Nettobarwert der Investition bei einem Zinssatz von 5 \% p.a.?
    \item Wie hoch ist der Nettobarwert der Investition bei einem Zinssatz von 10 \% p.a.?
    \item Lohnt sich die Investition wirtschaftlich bei einem Zinssatz von 5 \% p.a.? Lohnt sich die Investition wirtschaftlich bei einem Zinssatz von 10 \% p.a.?
\end{enumerate}

Verwenden Sie bei der Berechnung die jeweiligen Barwertfaktoren aus der obigen Tabelle.

\solution{
\textbf{Lösung: Nettobarwertmethode / Berechnung des Nettobarwerts}

\textbf{Gegeben:} Die erwarteten Ein- und Auszahlungen sowie die Barwertfaktoren für die nächsten fünf Jahre:

\begin{table}[H]
\centering
\begin{tabular}{|c|r|r|r|r|r|r|}
\hline
\textbf{Jahr} & \textbf{2023} & \textbf{2024} & \textbf{2025} & \textbf{2026} & \textbf{2027} & \textbf{2028} \\
\hline
\textbf{Betrag (€)} & -200.000 & 37.000 & 50.000 & 56.000 & 60.000 & 64.000 \\
\hline
\textbf{Barwertfaktor 5\%} & 1,00000 & 0,95238 & 0,90703 & 0,86384 & 0,82270 & 0,78353 \\
\hline
\textbf{Barwertfaktor 10\%} & 1,00000 & 0,90909 & 0,82645 & 0,75131 & 0,68301 & 0,62092 \\
\hline
\textbf{Present Value 5\% (€)} & -200.000 & 35.238,06 & 45.351,48 & 48.374,98 & 49.362,00 & 50.146,02 \\
\hline
\textbf{Present Value 10\% (€)} & -200.000 & 33.636,36 & 41.322,58 & 42.073,89 & 40.980,60 & 39.739,02 \\
\hline
\textbf{Kumulierter PV 5\% (€)} & -200.000 & -164.761,94 & -119.410,45 & -71.035,47 & -21.673,47 & +28.467,60 \\
\hline
\textbf{Kumulierter PV 10\% (€)} & -200.000 & -166.363,64 & -125.041,06 & -82.967,17 & -41.986,57 & -2.252,90 \\
\hline
\end{tabular}
\caption{Barwertfaktoren und Nettobarwerte bei 5\% und 10\% Zinssätzen.}
\end{table}

\begin{enumerate}[label=(\alph*)]
    \item \textbf{Nettobarwert bei 5\%:} \\
    Der kumulierte Nettobarwert bei einem Zinssatz von 5\% beträgt:
    \[
    \textbf{+28.467,60\,€.}
    \]

    \item \textbf{Nettobarwert bei 10\%:} \\
    Der kumulierte Nettobarwert bei einem Zinssatz von 10\% beträgt:
    \[
    \textbf{-2.252,90\,€.}
    \]

    \item \textbf{Wirtschaftlichkeit der Investition:} \\
    \begin{itemize}
        \item Die Investition lohnt sich bei einem Zinssatz von 5\%, da der kumulierte Nettobarwert positiv ist (\textbf{+28.467,60 €}).
        \item Die Investition lohnt sich \textbf{nicht} bei einem Zinssatz von 10\%, da der kumulierte Nettobarwert negativ ist (\textbf{-2.252,90 €}).
    \end{itemize}
\end{enumerate}
}


