\exercise{3\textbar1}{Produktionstypen}
Wichtige Einflüsse auf die zeitliche Verteilung der Produktionsmenge ergeben sich sowohl aus der auftrags- als auch der vorratsbezogenen Fertigung.

\begin{enumerate}[label=\alph*)]
    \item Beschreiben Sie die Vor- und Nachteile der auftrags- bzw. vorratsbezogenen Fertigung.
    \item Warum trifft man in der Praxis selten auf die reine Form der auftrags- bzw. der vorratsbezogenen Fertigung?
\end{enumerate}


\solution{
\begin{enumerate}[label=\alph*)]
    \item 
    Bei der vorratsbezogenen Fertigung produziert das Unternehmen aufgrund prognostizierter Absatzmengen auf Vorrat (marktorientiert). \\
    Die Vor- und Nachteile der vorratsbezogenen Fertigung verhalten sich spiegelbildlich zu den Vor- und Nachteilen der auftragsbezogenen Fertigung.
    
    \item 
    In der Praxis findet man häufig \textbf{Gemischtfertigung}, da Unternehmen versuchen, die Vorteile der auftragsbezogenen Fertigung mit denjenigen der vorratsbezogenen Fertigung zu kombinieren. \\
    Auch kann aufgrund des Produktionsprogramms des Unternehmens eine Kombination beider Fertigungsmethoden von Vorteil sein. \\
    Produkte der Einzelfertigung und natürlich auch Dienstleistungen werden zumeist auftragsbezogen hergestellt, während Massenware oft auf Vorrat produziert wird.
\end{enumerate}
}

\exercise{3\textbar2}{Fertigungstypen}{
Beschreiben Sie die Vor- und Nachteile der Einzel- und Massenfertigung.
}

\solution{
\textbf{Einzelfertigung:} \\
Die Einzelfertigung beruht nicht auf einem festen Produktionsprogramm, sondern es kommt ständig zu auftragsbezogenen Anpassungen von Produktionsanlagen, Arbeitskräften und Know-how an veränderte Nachfragebedingungen.

\begin{itemize}
    \item[+] Ein Vorteil der Einzelfertigung besteht im hohen Qualitätsniveau und dem technischen Know-how.
    \item[+] Für den entstandenen Mehrwert ist der Kunde bereit, mehr zu zahlen.
    \item[+] Ebenfalls von Vorteil ist die hohe Flexibilität eines Unternehmens mit Einzelfertigung, immer neue Produktvariationen zu entwickeln.
    \item[+] Tiefe Zins- und Lagerkosten sowie die einfache Erfassung des Bedarfs sind ebenfalls von Vorteil.
    \item[-] Nachteilig an der Einzelfertigung ist das erforderliche hohe Ausbildungsniveau der Mitarbeiter sowie das für Innovationen benötigte Knowhow.
    \item[-] Für eine ständige Umgestaltung der Produkte sind spezifische Investitionen erforderlich, die durch einen entsprechend hohen Verkaufspreis der Güter abgegolten werden müssen.
    \item[-] Weiterhin ist die Einzelfertigung kapitalintensiv (Spezialmaschinen, Fachpersonal) und bringt hohe Herstellungskosten mit sich.
\end{itemize}

\textbf{Massenfertigung:} \\
Die Massenfertigung zeichnet sich dadurch aus, dass von einem oder mehreren Produkten über längere Zeit sehr große Stückzahlen hergestellt werden.

\begin{itemize}
    \item[+] Ein Vorteil der Massenfertigung liegt darin, dass derselbe Fertigungsprozess ununterbrochen aufrechterhalten werden kann.
    \item[+] Kosten für die Veränderung der Fertigungsanlagen entfallen größtenteils.
    \item[+] Durch die geringen Wechsel in der Produktion ist eine weitgehende Automatisierung möglich, was die Fertigung sehr hoher Stückzahlen ermöglicht.
    \item[+] Hierdurch können die Herstellungskosten und somit der Verkaufspreis gesenkt werden.
    \item[+] Die Anforderungen an das Ausbildungsniveau der Mitarbeiter sind gering.
    \item[-] Nachteilig an der Massenfertigung ist die geringe Flexibilität bezüglich Veränderungen in der Nachfragestruktur.
    \item[-] Kommt es unerwartet zu einem Wandel der Nachfragepräferenzen, so erweisen sich spezifische Investitionen in Spezialmaschinen möglicherweise als unrentabel.
    \item[-] In der Massenfertigung kommt bereits bei kleinen Störungen der gesamte Produktionsprozess zum Stillstand, sodass Zwischenlager aufgebaut werden müssen.
    \item[-] Ferner entstehen meist hohe Zins- und Lagerkosten.
\end{itemize}
}

\exercise{3\textbar3}{Produktion}{
Marlene befindet sich heute in den Produktionshallen der OHM GmbH. Herr Eckert ist dort der Zuständige für die Basic-Line-Produktlinie. Marlene hat bereits einen guten Überblick über die Produktpalette erworben. Die Basic Line umfasst Kaffeevollautomaten für Büros und kleinere Betriebe. Die Coffee easy ist der kleinste Vollautomat, den die OHM GmbH anbietet. Er ist einfach zu bedienen und zu reinigen, für kleinere Betriebe und Büros geeignet. Daneben gibt es in der Basic Line noch den Vollautomaten Coffee Business. Dieser bietet sehr viel mehr Variationsmöglichkeiten, ist schrittweise ausbaubar und erweiterbar.
\\~\\
Herr Eckert: „Als erstes zeige ich Ihnen mal, wo wir unsere Waren lagern. Hier im Materiallager lagern einmal die Teile unserer Zulieferer. Diese haben Sie ja bereits zum Teil in unseren ABC-Analysen kennengelernt. Ein paar Regalreihen weiter lagern unsere Zwischenprodukte. Dies sind Bau- oder Maschinenteile, die wir hier produzieren und die nun auf ihren endgültigen Einbau warten. Das sind zum Beispiel unsere Mahlwerke sowie die Brüheinheiten. Wir haben eine jahrelange Erfahrung mit dem Kolbendruckverfahren. Deswegen produzieren unsere Maschinen auch diese hohe Kaffeequalität.“
\\~\\
Im Lager ist ein reges Treiben. Gerade fährt ein kleiner Transportwagen aus der Produktionshalle ins Lager. Herr Eckert zieht Marlene etwas zur Seite. „Aufpassen, wir haben hier Vorfahrtsregeln. Bleiben Sie besser auf dem Fußgängerweg.“ Jetzt versteht Marlene die Linien, die am Boden aufgezeichnet sind. Diese geben wohl die Transportwege vom Lager zur Produktfertigung vor. Und dann gibt es noch einen roten breiten Weg, der wohl für die Fußgänger frei ist. „Schauen Sie mal, Frau Schuberth. Der Wagen, der gerade an uns vorbeikam, bringt fertige Bauteile ins Lager. Das ist der Wagen aus der Coffee Business-Linie. Er hat die Basis-Maschinen geladen. Auch das sind bei uns Zwischenprodukte, die hier lagern und je nach Kundenbestellung zur Endfertigung kommen.“
\\~\\
Marlene kennt bereits das Konfigurationssystem der OHM GmbH. Der Kunde bestellt immer zuerst eine Basis-Maschine und wählt dann dazu seine Erweiterungskomponenten. „So, und dann lassen Sie uns doch mal weiter in die Fertigung gehen. Wir haben hier keine Fließbandfertigung. Wir fertigen in sogenannten Gruppen. Jede Gruppe stellt ein Bauteil her, die Zwischenprodukte, die ich Ihnen gerade gezeigt habe.“
\\
„Und wo findet denn die Endfertigung statt?“, fragt Marlene neugierig.
\\
„Das zeige ich Ihnen später. Die Endfertigung ist in der Halle gegenüber. Da kommen dann je nach Kundenanforderungen alle Maschinenteile zusammen. Dort wird auch die kundenspezifische Programmierung und die Qualitätskontrolle vorgenommen.“
\\~\\
Einer unserer Lieferanten ist die Firma IXO. Diese stellt Schläuche und Rohrverbindungen her. Sie hat sich vor allem auf Geräte mit kompakter Bauweise spezialisiert. Also Verbindungen für Maschinen mit engen Platzverhältnissen. Wir haben eine sehr lange Geschäftsverbindung mit IXO und ein sehr großes Vertrauensverhältnis. Jeweils Donnerstag werden Behälter mit den Teilen geliefert und mit leeren Behältern, die bei uns geblieben sind, ausgetauscht. Die Firma IXO nimmt die leeren Behälter mit und tauscht sie gegen gefüllte Behälter aus. Sollte ein Artikel mal außerplanmäßig vor dem Austauschtag ausgehen, kann jederzeit bei IXO geordert werden. Dies wollen wir aber möglichst vermeiden, deswegen auch unsere ständigen Statistiken über den Materialverbrauch.

\begin{enumerate}[label=\alph*)]
    \item Welche Produktionsstrategie verfolgt die OHM GmbH?
    \item Welchen Produktionstyp hat die OHM GmbH gewählt?
    \item Welchem Fertigungstyp entspricht wohl die Produktion der OHM GmbH?
    \item Um welchen Gütertyp handelt es sich hier bei den Teilen des Zulieferers IXO?
\end{enumerate}
}

\solution{
\textbf{a) Produktionsstrategie} \\
Die OHM GmbH verfolgt eine \textbf{Diversifikationsstrategie}. Dies wird durch die Möglichkeit der kundenspezifischen Anpassung der Produkte deutlich. Die Basis-Maschinen können je nach Kundenanforderung erweitert werden. Ziel ist es, eine hohe Variantenvielfalt und Flexibilität in der Produktion anzubieten.

\textbf{b) Produktionstyp} \\
Die OHM GmbH nutzt eine \textbf{Mischfertigung}. Die Basis-Maschinen werden unabhängig von spezifischen Kundenaufträgen vorgefertigt (vorratsbezogene Fertigung). Die Endfertigung erfolgt jedoch kundenspezifisch (auftragsbezogene Fertigung), indem die Basis-Maschinen mit Erweiterungskomponenten kombiniert und programmiert werden.

\textbf{c) Fertigungstyp} \\
Die Produktion der OHM GmbH entspricht einer \textbf{Serienfertigung} in Kombination mit Elementen der \textbf{Mass Customization}. Die Basis-Maschinen werden in größeren Stückzahlen vorgefertigt (Serienfertigung). Die Endfertigung erfolgt individuell nach Kundenwunsch, was eine Anpassung an spezifische Anforderungen ermöglicht (Mass Customization).

\textbf{d) Gütertyp der Teile des Zulieferers IXO} \\
Die Teile des Zulieferers IXO (Schläuche und Rohrverbindungen) fallen unter die Kategorie der \textbf{Produktionsgüter}, genauer gesagt der \textbf{Werkstoffe}. 

Diese Werkstoffe werden in der Produktion der OHM GmbH als direkte Inputfaktoren verwendet, da sie in die herzustellenden Produkte eingehen. Aufgrund der engen Platzverhältnisse, auf die IXO spezialisiert ist, handelt es sich um \textbf{Spezialgüter}, die eine hohe Spezialisierung und Qualität erfordern. Zudem deutet die Just-in-Time-Lieferung und das Behältermanagement darauf hin, dass es sich um \textbf{verbrauchsorientierte Güter} handelt, die regelmäßig nach Bedarf geliefert werden.

}

\exercise{3|4}{Break-even-Analyse}{
Marlene musste sich heute bei Herrn Peters in der Rechnungswesen-Abteilung melden. Dieser informiert sie über die
aktuellen Tagesaufgaben. Marlene soll aus den variablen und fixen Kosten in der Produktion die Break-even-Menge
und den Break-even-Umsatz für die Coffee Easy berechnen.
\\~\\
Da Marlene noch aus der Vorlesung „Einführung in die Allgemeine BWL“ weiß, wie die Berechnung im Einzelnen
erfolgt, legt sie sofort los und versucht zunächst die aktuellen Werte für die variablen und fixen Kosten der Coffee
Easy herauszufinden.
\\~\\
Der Mitarbeiter, Herr Hugo, hilft ihr und bringt ihr die entsprechenden Unterlagen, aus denen die Kosten ersichtlich
sind: Marlene kann sehen, dass die monatliche Miete für die Produktionshalle bei 2.000 € liegt. Weiterhin sieht sie,
dass sich die Materialkosten auf 300 € pro Stück belaufen. Die Personalkosten machen im Monat 3.000 € aus. Für die
Herstellkosten fallen 150 € pro Stück an. Marlene erinnert sich, dass die Vertriebskosten ebenfalls mit eine Rolle
spielen und ruft daher bei der Vertriebsabteilung an, um die Kosten pro Einheit zu erfahren. Der Vertriebsleiter, Herr
Cuva, teilt ihr mit, dass die Kosten bei 50 € liegen.
\\~\\
Der Verkaufspreis pro Stück liegt bei 1.000 €. Herr Peters erwägt jedoch, den Preis auf 800 € zu senken, da der Absatz
der Coffee Easy im Vergleich zu den anderen Produkten der OHM GmbH stagniert.
\\~\\
\textbf{a)} Wie ändert sich dann die Break-even-Menge und der Break-even-Umsatz?\\
\textbf{a)} Würden Sie Herrn Peters eine Senkung des Verkaufspreises raten?
\\~\\
\note{Helfen Sie Marlene und ermitteln Sie den mengen- und wertmäßigen Break-even!}
}
\solution{
\textbf{Berechnung der Break-even-Menge und des Break-even-Umsatzes:}

\textbf{Berechnung bei einem Verkaufspreis von 1.000 €:}
\[
Break-even (Menge) = \frac{Summe der Fixkosten}{Preis pro Stück - Variable Kosten pro Stück}\]
\[
= \frac{5.000\,€}{1.000\,€ - (300\,€ + 150\,€ + 50\,€)} = \frac{5.000\,€}{500\,€} = 10 \,Stück
\]

\[
Break-even (wertmäßig) = \frac{Summe der Fixkosten}{Deckungsquote je Stück} =
\frac{5.000\,€}{\frac{500\,€}{1.000\,€}} = \frac{5.000\,€}{0.5} = 10.000\,€
\]

\textbf{Berechnung bei einem Verkaufspreis von 800 €:}
\[
Break-even (Menge) = \frac{Summe der Fixkosten}{Preis pro Stück - Variable Kosten pro Stück} \]
\[
= \frac{5.000\,€}{800\,€ - (300\,€ + 150\,€ + 50\,€)} = \frac{5.000\,€}{300\,€} = 16,67 \,Stück \approx 17 \,Stück
\]

\[
Break-even (wertmäßig) = \frac{Summe der Fixkosten}{Deckungsquote je Stück} = \frac{5.000\,€}{\frac{300\,€}{800\,€}} = \frac{5.000\,€}{0.375} = 13.333,33\,€
\]

\textbf{Antwort auf b):}

Eine Senkung des Verkaufspreises auf 800 € führt zu einer deutlichen Erhöhung der Break-even-Menge von 10 auf 17 Stück. Das bedeutet, dass die OHM GmbH deutlich mehr Einheiten verkaufen müsste, um ihre Fixkosten zu decken. Zudem sinkt der wertmäßige Break-even von 10.000 € auf 13.333,33 €, was die Profitabilität weiter belastet.

\textbf{Empfehlung:} 
Eine Preissenkung sollte nur dann in Betracht gezogen werden, wenn:
\begin{itemize}
    \item der Marktanteil durch die Senkung deutlich gesteigert werden kann,
    \item die fixen Kosten durch erhöhte Absatzmengen besser gedeckt werden,
    \item keine stärkeren Einbußen bei der Deckungsbeitragsquote auftreten.
\end{itemize}
Andernfalls wird die Rentabilität der Coffee Easy weiter beeinträchtigt, und eine Preissenkung wäre nicht zu empfehlen.
}

\exercise{3\textbar5}{Kosten, Erlöse, Gewinn}{
Ein Unternehmen verkauft Batterien für Digitalkameras und hat folgende Kosten- und Erlössituation:
\begin{itemize}
    \item Fixkosten: 5.300 €
    \item Variable Kosten pro Stück: 4,20 €
    \item Verkaufspreis pro Stück: 9,50 €
\end{itemize}
Es werden monatlich 5.000 Stück verkauft. Die Kostenfunktion ist linear.
\\~\\
Berechnen Sie bitte …
\begin{enumerate}[label=\alph*)]
    \item die Gesamtkosten,
    \item den Erlös,
    \item den Gewinn,
    \item und die Gewinnschwelle.
\end{enumerate}
}

\solution{
\textbf{Lösung der Aufgabe 3|5: Kosten, Erlöse, Gewinn:}

\textbf{a) Gesamtkosten:}
\[
K(M) = K_f + (M \cdot K_v) = 5.300 \,€ + (5.000 \cdot 4,20 \,€) = 26.300 \,€
\]

\textbf{b) Erlös:}
\[
E = p \cdot M = 9,50 \,€ \cdot 5.000 = 47.500 \,€
\]

\textbf{c) Gewinn:}
\[
G(M) = E(M) - K(M) = 47.500 \,€ - 26.300 \,€ = 21.200 \,€
\]

\textbf{d) Gewinnschwelle:}
Die Gewinnschwelle ist die Menge, bei der der Gewinn 0 ist:
\[
G(M) = E(M) - K(M) = 9,50 \,€ \cdot M - (5.300 \,€ + (M \cdot 4,20 \,€))
\]
\[
9,50 \,€ \cdot M - (4,20 \,€ \cdot M + 5.300 \,€) = 0
\]
\[
5,30 \,€ \cdot M = 5.300 \,€
\]
Aufgelöst nach \(M\):
\[
M = \frac{5.300 \,€}{5,30 \,€} = 1.000 \,Stück
\]
}

\exercise{3|6}{Break-even-Analyse}{
Ein Unternehmen erzielt für sein Produkt einen Preis von 20 € pro Stück.\\
Die variablen Kosten pro Stück betragen 10 €.\\
Die fixen Kosten pro Jahr belaufen sich auf 40.000 €.\\

\begin{enumerate}[label=\alph*)]
    \item Bei welcher Absatzmenge erreicht das Unternehmen die Gewinnschwelle?
    \item Wie hoch ist der Gewinn bei einer Absatzmenge von 6.000 Stück?
    \item Das Unternehmen möchte mit einem Budget \(W\) von 10.000 € eine Werbeaktion durchführen. 
    Um wie viel muss sich die Absatzmenge gegenüber dem Ergebnis aus Teilaufgabe b) erhöhen, 
    wenn der dort erzielte Gewinn wieder erreicht werden soll?
\end{enumerate}
}

\solution{

\textbf{a)}
\[
Break-even (Menge) = \frac{Summe der Fixkosten}{Preis pro Stück - variable Kosten pro Stück} = \frac{40.000 \,€ }{20 \,€  - 10 \,€ } = 4.000 \,Stück
\]

\textbf{b) }
\[
Gewinn = Erlös - variable Kosten - Fixkosten
\]
\[
Gewinn = (6.000 \cdot 20 \,€ ) - (6.000 \cdot 10 \,€ ) - 40.000 \,€  = 20.000 \,€ 
\]

\textbf{c) }
\[
Gewinn = M \cdot (p - k_v) - k_f - W
\]
\[
20.000 = M \cdot (20 \,€ - 10 \,€ ) - 40.000 \,€  - 10.000 \,€ 
\]
\[
20.000 = M \cdot 10 - 50.000
\]
\[
M \cdot 10 = 20.000 + 50.000
\]
\[
M = \frac{70.000}{10} = 7.000 \,Stück
\]
\textbf{Die Absatzmenge muss sich also um mindestens 1.000 Stück erhöhen, damit sich die Werbeaktion lohnt.}
}

\exercise{3|7}{Break-even-Analyse}{
Ein Festplattenhersteller plant die Herstellung und den Verkauf einer neuen externen USB-Festplatte.
Für die Herstellung und den Verkauf dieser Festplatte werden für den Hersteller verschiedene
Kostenpositionen anfallen: Die monatlichen Personalkosten betragen 9.500,00 €. Die monatliche Miete für
die Produktionshalle beträgt 3.500,00 €. Weiterhin fallen je Festplatte Materialkosten in Höhe von 3,00 € an.
Für den Produktionsvorgang fallen zusätzlich 0,50 € pro Festplatte an. Zudem ist mit Vertriebskosten in
Höhe von 0,30 € je Festplatte zu rechnen.

Als Verkaufspreis für die Festplatte werden 20,00 € geplant.

\textbf{Aufgaben:}
\begin{itemize}
    \item[a)] Ermitteln Sie den mengenmäßigen Break-even.
    \item[b)] Ermitteln Sie den wertmäßigen Break-even.
\end{itemize}

\note{Runden Sie die berechneten Endergebnisse auf 2 Stellen nach dem Komma.}
}

\solution{

\textbf{a)}
\[
Break-even (Menge) = \frac{Summe der Fixkosten}{Preis pro Stück - variable Kosten pro Stück}
\]
\[
Break-even (Menge) = \frac{9.500 \,€ + 3.500 \,€}{20 \,€ - (3 \,€ + 0,50 \,€ + 0,30 \,€)} = \frac{13.000 \,€}{16,20 \,€} \approx 802,47 \,Stück
\]

\textbf{b) }
\[
Break-even (Wert) = Break-even (Menge) \cdot Preis pro Stück
\]
\[
Break-even (Wert) = 802,47 \cdot 20,00 \,€ \approx 16.049,38 \,€
\]
}

\exercise{3|8}{Break-even-Analyse und Deckungsbeitrag}{
Die Aristo AG produziert und vertreibt zwei verschiedene Produkte. Aus den Abteilungen Produktion und Marketing sind folgende Zahlen bekannt:

\begin{table}[h!]
\centering
\begin{tabular}{|l|c|c|}
\hline
\textbf{}                     & \textbf{Produkt P\textsubscript{1}} & \textbf{Produkt P\textsubscript{2}} \\ \hline
Verkaufspreis/Einheit (€)     & 90                                 & 318                                \\ \hline
\textbf{Materialbedarf (kg):} &                                     &                                     \\ \hline
Material A                    & 5                                   & 10                                  \\ \hline
Material B                    & 2                                   & 3                                   \\ \hline
Material C                    & -                                   & 5                                   \\ \hline
Lohnkosten/Einheit (€)        & 20                                  & 130                                 \\ \hline
Maschinenkosten/Einheit (€)   & 10                                  & 120                                 \\ \hline
Fixkosten (€)                 & unbekannt                           & 4.800                               \\ \hline
\textbf{Produktionszeit/Einheit (Stunden):} &                         &                                     \\ \hline
Maschine 1                    & 4                                   & 10                                  \\ \hline
Maschine 2                    & 12                                  & 6                                   \\ \hline
Maschine 3                    & 6                                   & -                                   \\ \hline
\end{tabular}
\end{table}
\noindent
Der Einkaufspreis pro Kilogramm für die Materialarten beträgt:
\begin{itemize}
    \item Material A: 2 €
    \item Material B: 10 €
    \item Material C: 2 €
\end{itemize}
\noindent
Die Maximalkapazitäten der Maschinen betragen:
\begin{itemize}
    \item Maschine 1: 18.000 Stunden
    \item Maschine 2: 24.000 Stunden
    \item Maschine 3: 12.000 Stunden
\end{itemize}
\noindent
\textbf{Aufgaben:}
\begin{enumerate}[label=\alph*)]
    \item Bei wie vielen Einheiten liegt die Nutzschwelle (Break-even-Punkt) für Produkt P\textsubscript{2}?
    \item Wie groß ist der Deckungsbeitrag der einzelnen Produkte?
\end{enumerate}
}

\solution{

\textbf{a) }

\begin{itemize}
    \item \textbf{Daten für das Produkt P\textsubscript{2}:}
    \begin{itemize}
        \item \textbf{Variable Kosten (k\textsubscript{var}):} 310 € (Material: 60 €, Löhne: 130 €, Maschinen: 120 €)
        \item \textbf{Fixkosten (K\textsubscript{fix}):} 4.800 €
        \item \textbf{Verkaufspreis (p):} 318 €
    \end{itemize}
    \item \textbf{Berechnung:} Break-even-Punkt: Erlös = Kosten
    \[
    p \cdot x = k\textsubscript{var} \cdot x + K\textsubscript{fix}
    \]
    \[
    318 \cdot x = 310 \cdot x + 4.800
    \]
    \[
    8 \cdot x = 4.800
    \]
    \[
    x = \frac{4.800}{8} = 600
    \]
    \item \textbf{Der Break-even-Punkt liegt bei 600 Einheiten.}
\end{itemize}

\textbf{b) }

\begin{table}[h!]
\centering
\begin{tabular}{|l|c|c|}
\hline
\textbf{}                     & \textbf{Produkt P\textsubscript{1}} & \textbf{Produkt P\textsubscript{2}} \\ \hline
Verkaufserlös (€)             & 90                                  & 318                                 \\ \hline
Materialkosten A (€)          & 10                                  & 20                                  \\ \hline
Materialkosten B (€)          & 20                                  & 30                                  \\ \hline
Materialkosten C (€)          & -                                   & 10                                  \\ \hline
Lohnkosten (€)                & 20                                  & 130                                 \\ \hline
Maschinenkosten (€)           & 10                                  & 120                                 \\ \hline
\textbf{Deckungsbeitrag (€)}  & 30                                  & 8                                   \\ \hline
\end{tabular}
\end{table}
}

